%\PassOptionsToPackage{x11names}{xcolor}
\documentclass[11pt]{article}
%\usepackage{beamerarticle}
%\usepackage{xunicode}
%\usepackage{fontspec}
%\usepackage{xltxtra}
\usepackage{booktabs}
\usepackage{graphicx}
\usepackage{multirow}
\usepackage{gb4e}
\usepackage{polyglossia}
%\usepackage{tikz}
%\usetikzlibrary{decorations.fractals,fit,shapes.geometric}
%\usepackage{newicktree}
\usepackage{natbib}
\bibpunct[:]{(}{)}{,}{a}{}{,}
\usepackage{hyperref}

\setdefaultlanguage{english}
\setmainfont[Mapping=tex-text,Ligatures=Common]{Linux Libertine O}
\newfontfamily\phon[Mapping=tex-text,Ligatures=Common,Scale=MatchLowercase]{Charis SIL} 
\newfontfamily\phondroit[Mapping=tex-text,Ligatures=Common,Scale=MatchLowercase]{Charis SIL} 

\newfontfamily\cn[Mapping=tex-text]{WenQuanYi Micro Hei} 
\newcommand{\zh}[1]{{\cn #1}}
\XeTeXlinebreaklocale 'zh' %使用中文换行
\XeTeXlinebreakskip = 0pt plus 1pt %

% \usetheme{Torino}
% \usefonttheme{serif}
% \setbeamerfont{title like}{shape=\scshape}
% \setbeamerfont{frametitle}{shape=\scshape}
% \setbeamertemplate{frametitle continuation}[from second]
% \setbeamertemplate{footline}[frame number]
% \setbeamertemplate{navigation symbols}{}


\newcommand{\ipa}[1]{{\textit{\phon #1}}} %API tjs en italique
\newcommand{\ipapl}[1]{{\phondroit #1}} 
\newcommand{\racine}[1]{\begin{math}\sqrt{#1}\end{math}} 
\newcommand{\grise}[1]{\cellcolor{lightgray}\textbf{#1}} 
\usepackage{epsf}
\newcommand{\ra}{$\Sigma_1$} 
\newcommand{\rc}{$\Sigma_3$} 
\newcommand{\ro}{$\Sigma$} 
% \newcommand{\rouge}[1]{{\color{red}#1}}
% \newcommand{\bleu}[1]{{\color{blue}#1}}

\title{Semi-direct speech in Rgyalrongic languages: Rtau vs Japhug} 

%\author[shortname]{Anton Antonov \inst{1} \and Guillaume Jacques \inst{2}}
%\institute[shortinst]{\inst{1}INALCO-CRLAO \and %
                      %\inst{2} CRLAO-EHESS-INALCO-CNRS}
%\institute{\textsc{Crlao--Inalco--Cnrs--Ehess}}

\author{Anton Antonov (INALCO-CRLAO)  \& Guillaume Jacques (CRLAO-EHESS-INALCO-CNRS)}

%\newcommand{\bipa}[1]{{\LARGE\phon #1}}
\newcommand{\subone}{₁}
\newcommand{\subtwo}{₂}
\newcommand{\agent}{\textsc{a}}
\newcommand{\abl}{\textsc{abl}}
\newcommand{\abs}{\textsc{abs}}
\newcommand{\acc}{\textsc{acc}}
\newcommand{\addr}{\textsc{addr}}
\newcommand{\adn}{\textsc{adn}}
\newcommand{\allat}{\textsc{all}}
\newcommand{\alloc}{\textsc{alloc}}
\newcommand{\art}{\textsc{art}}
\newcommand{\asp}{\textsc{asp}}
\newcommand{\assert}{\textsc{decl}}
\newcommand{\aor}{\textsc{aor}}
\newcommand{\auth}{\textsc{auth}}
\newcommand{\aux}{\textsc{aux}}
\newcommand{\cnv}{\textsc{cnv}}
\newcommand{\comp}{\textsc{comp}}
\newcommand{\concess}{\textsc{concess}}
\newcommand{\cond}{\textsc{cond}}
\newcommand{\conv}{\textsc{cvb}}
\newcommand{\const}{\textsc{testim}}
\newcommand{\cop}{\textsc{cop}}
\newcommand{\dest}{\textsc{dest}}
\newcommand{\dat}{\textsc{dat}}
\newcommand{\deb}{\textsc{deb}}
\newcommand{\defin}{\textsc{def}}
\newcommand{\defer}{\textsc{defer}}
\newcommand{\dem}{\textsc{dem}}
\newcommand{\dep}{\textsc{dep}}
\newcommand{\determ}{\textsc{det}}
\newcommand{\elat}{\textsc{el}}
\newcommand{\emphat}{\textsc{emph}}
\newcommand{\equat}{\textsc{equat}}
\newcommand{\erg}{\textsc{erg}}
\newcommand{\evid}{\textsc{evid}}
\newcommand{\excl}{\textsc{excl}}
\newcommand{\exclam}{\textsc{exclam}}
\newcommand{\fem}{\textsc{f}}
\newcommand{\familier}{\textsc{fam}}
\newcommand{\fin}{\textsc{fin}}
\newcommand{\foc}{\textsc{foc}}
\newcommand{\fut}{\textsc{fut}}
\newcommand{\gen}{\textsc{gen}}
\newcommand{\ger}{\textsc{ger}}
\newcommand{\hon}{\textsc{hon}}
\newcommand{\hort}{\textsc{hort}}
\newcommand{\hum}{\textsc{hum}}
\newcommand{\hyp}{\textsc{hyp}}
\newcommand{\imp}{\textsc{imp}}
\newcommand{\imprf}{\textsc{imprf}}
\newcommand{\inch}{\textsc{incho}}
\newcommand{\incl}{\textsc{incl}}
\newcommand{\ind}{\textsc{ind}}
\newcommand{\indef}{\textsc{indef}}
\newcommand{\infin}{\textsc{inf}}
\newcommand{\instr}{\textsc{instr}}
\newcommand{\inter}{\textsc{inter}}
\newcommand{\inv}{\textsc{inv}}
\newcommand{\loc}{\textsc{loc}}
\newcommand{\masc}{\textsc{m}}
%\newcommand{\mod}{\textsc{mod}}
\newcommand{\negate}{\textsc{neg}}
\newcommand{\neut}{\textsc{n}}
\newcommand{\nmlz}{\textsc{nmlz}}
\newcommand{\nom}{\textsc{nom}}
\newcommand{\oblique}{\textsc{obl}}
\newcommand{\opt}{\textsc{opt}}
\newcommand{\partit}{\textsc{part}}
\newcommand{\pat}{\textsc{p}}
\newcommand{\pl}{\textsc{pl}}
\newcommand{\poss}{\textsc{poss}}
\newcommand{\pot}{\textsc{pot}}
\newcommand{\prev}{\textsc{pv}}
\newcommand{\prf}{\textsc{prf}}
\newcommand{\prox}{\textsc{prox}}
\newcommand{\prs}{\textsc{prs}}
\newcommand{\prt}{\textsc{prt}}
\newcommand{\pass}{\textsc{pass}}
\newcommand{\past}{\textsc{pst}}
\newcommand{\pref}{\textsc{pref}}
\newcommand{\prog}{\textsc{prog}}
\newcommand{\ptcp}{\textsc{ptcp}}
\newcommand{\quot}{\textsc{quot}}
%\newcommand{\rad}{\textsc{rad}}
\newcommand{\refl}{\textsc{refl}}
\newcommand{\rel}{\textsc{rel}}
\newcommand{\resp}{\textsc{rsp}}
\newcommand{\seq}{\textsc{seq}}
\newcommand{\sg}{\textsc{sg}}
\newcommand{\simil}{\textsc{simil}}
\newcommand{\subj}{\textsc{subj}}
\newcommand{\super}{\textsc{super}}
\newcommand{\topic}{\textsc{top}}
\newcommand{\tranz}{\textsc{tr}}
\newcommand{\unique}{\textsc{s}}
\newcommand{\voc}{\textsc{voc}}
%%%%%%%%%%%%%%%%%%%%%%%%%%%%%%%%%%%%%%
%\newcommand{\:}{{:}}
 %environnement \gloss{xxx} en petites capitales
%\newcommand{\ipa}[1]{{\phon #1}}%environnement \ipa{xxx} pour la phonétique
%\newcommand{\lng}[1]{{\phon\itshape #1}}%environnement \lng{xxx} pour la phonétique en italique

\begin{document} 
\maketitle



\section{Semi-direct speech in Rtau}

\subsection{The Rtau language}


\begin{itemize}
 \item  Rtau (locally known as \ipa{rəsɲəske}) is a Rgyalrongic language spoken in Rtau county (\zh{道孚县} Dàofú xiàn), Sichuan province (\zh{四川省} Sìchuān shěng), China.

\includegraphics[width=0.5\textwidth]{chinalingmap.jpg}\includegraphics[width=0.5\textwidth]{dawu.png}

\end{itemize}
 

\begin{itemize}
 \item Previous work on Rtau includes \citet{huangbf91daofu}, \citet{jackson07shangzhai} and especially \citet{sun13gexi}.
\item The variety presented here represents the dialect of Khang.gsar (\ipa{qʰərŋe}) village spoken in the North of Rtau county, and differs slightly from the varieties studied by other authors.

\item The  data presented here is based on ongoing fieldwork by the authors.

%\item 

\end{itemize}



% 
%  \begin{newicktree}
% % \small
% 
% \righttree \nobranchlengths \setunitlength{10cm} \nodelabelformat{}
% \drawtree[R]{((((((Resnyeske:0.07,Dge-bshes:0.07,Stod-sde:0.07):0.1[Trehor],(Thugs-chen:0.07,Njo-rogs:0.07):0.1[Lavrung]):0.05,(Japhug:0.07, Tshobdun:0.07, Showu:0.07,(N. Situ:0.02,S. Situ:0.02):0.05[Situ]):0.15[Rgyalrong]):0.12[Rgyalrongique],(Tangoute:0.2,Pumi:0.2):0.14[Pumi-Tangut],(N. Qiang:0.07,S. Qiang:0.07):0.27[Qiang],Queyu:0.34,Zhaba:0.34,Muya:0.34):0.08[MR],((Naxi:0.07,Na:0.07,Laze:0.07):0.1[Naish],Shixing:0.17,Namuyi:0.17):0.25[Naic],(Ersu:0.07,Lizu:0.07,Tosu:0.07):0.35[Ersuic]):0.05,Lolo-Burmese:0.05):0.05[BQ];}
%  \\~\\\scalebar[0.1]
% \end{newicktree}
% 
%[shrink=10]
\subsection{Person marking and flagging}

\begin{exe}
\ex \label{ex:np3}
\gll
	\ipa{tʂaɕi-w} \ipapl{dʐʚma} 	\ipapl{de} \ipapl{nə-f-se(-sə)} \\
	{Bkrashis-\erg} Sgrolma {\dem} {\prf-\inv-kill(-\evid)}\\ 
	\glt Bkrashis killed Sgrolma.
\end{exe}

\begin{exe}
\ex \label{ex:33}
\gll
	\ipa{tə-w} \ipapl{dʐʚma} \ipapl{de} \ipapl{nə-f-se(-sə)} \\
	{3\sg-\erg} Sgrolma {\dem} {\prf-\inv-kill(-\evid)}\\ 
	\glt He killed Sgrolma.
\end{exe}

\begin{exe}
\ex \label{ex:13}
\gll
	\ipa{ŋa}	\ipapl{dʐʚma} 	\ipapl{de} 	\ipapl{nə-se-w} \\
	{1\sg} Sgrolma {\dem} {\prf-kill-1\sg>3}\\ 
	\glt I killed Sgrolma.
\end{exe}

\begin{exe}
\ex \label{ex:23}
\gll
	\ipa{ɲi}	\ipapl{dʐʚma} 	\ipapl{de} 	\ipapl{nə-se-j(-sə)} \\
	{2\sg} Sgrolma {\dem} {\prf-kill-2\sg>3(-\evid)}\\ 
	\glt You killed Sgrolma.
\end{exe}




 
\begin{itemize}
 \item It is important to note that contrary to what happens in some other Sino-Tibetan languages (esp. in Tibetan languages and some other members of the Rgyalrongic branch) ergative marking is normally blocked on \textsc{sap}.%, with \textsc{sds} being the only exception as far as the first person singular is concerned (the second person singular pronoun remains unmarked as usual as in \ref{ex:1ebis}).
\end{itemize}



\section{Defining the topic}

%[shrink=15]
\subsection{Semi-direct speech}

\begin{itemize}

\item In Rtau, there are two main ways of reporting what someone else has said, direct speech (\textsc{ds}) and semi-direct speech (\textsc{sds}, cf. \citealp{aikhenvald08semidirect}, also called `hybrid speech' by \citealp{tournadre08conjunct}). 

%\item Unlike direct speech, \textsc{sds} is framed by a converbal form and an appropriately conjugated form of the reporting verb \ipa{jə} `say'

\item We have been unable to ascertain the existence of any other type of reported speech, such as indirect speech (\textsc{is}) which is reported to be unattested in Tibetan languages as well (\citealp{tournadre08conjunct}).

\end{itemize}



%[shrink=10]
\subsection{Semi-direct speech}

\begin{itemize}
 \item \textsc{sds} in Rtau is of type II in \cite{aikhenvald08semidirect}'s terms, i.e. the current speaker (\textsc{cs}) reports an original speaker (\textsc{os})'s words (or thoughts) by partly adjusting them to their (\textsc{cs}) perspective. 

\end{itemize}

\begin{exe}
\ex \label{ex:1b}
\gll
	\ipa{tʂaɕi-w}$_i$ \ipa{ŋa-gi}	\ipa{jə-rə-ge} [\ipapl{tə-w}$_j$	\ipapl{dʐʚma} 	\ipapl{de} \ipapl{nə-f-se-sə}] \ipa{jə-rə}  \\
	{Bkrashis-\erg} {1\sg-\dat} {say-\const-\conv} {3\sg-\erg} Sgrolma {\dem} {\prf-\inv-kill-\evid}  say-\const\\ 
	\glt Bkrashis$_i$ told me (that) he$_j$ ($\ne$ Bkrashis) had killed Sgrolma.
\end{exe}

\begin{exe}
\ex \label{ex:1c}
\gll
	\ipa{tʂaɕi-w}$_i$ \ipa{ŋa-gi}	\ipa{jə-rə-ge} [\ipapl{ədə}$_i$	\ipapl{dʐʚma} 	\ipapl{de} 	\ipapl{nə-se-w}] \ipa{jə-rə}  \\
	{Bkrashis-\erg} {1\sg-\dat} {say-\const-\conv} {\refl} Sgrolma {\dem} {\prf-kill-1\sg>3} say-\const\\ 
	\glt Bkrashis$_i$ told me that he$_i$ had killed Sgrolma.
\end{exe}



\begin{itemize}

 \item absence of coreference in case of a third person \textsc{os} is signalled by the use of the ordinary third person pronoun \ipa{tə} (cf. \ref{ex:1bbis})

\begin{exe}
\ex \label{ex:1bbis}
\gll
	\ipa{tʂaɕi-w}$_i$ \ipa{ŋa-gi}	\ipa{jə-rə-ge} [\ipapl{tə-w}$_j$	\ipapl{dʐʚma} 	\ipapl{de} \ipapl{nə-f-se-sə}] \ipa{jə-rə}  \\
	{Bkrashis-\erg} {1\sg-\dat} {say-\const-\conv} {3\sg-\erg} Sgrolma {\dem} {\prf-\inv-kill-\evid}  say-\const\\ 
	\glt Bkrashis$_i$ told me (that) he$_j$ ($\ne$ Bkrashis) had killed Sgrolma.
\end{exe}

\end{itemize}


\begin{itemize}
 \item coreference between the \textsc{os} as subject of the reporting (or cognitive) verb and (one of) the participant(s) in the reported speech is signalled by the (logophoric) use of the reflexive pronoun \ipa{ədə(qʰʚ)} `oneself' instead of the first person pronoun \ipa{ŋa} (cf. \ref{ex:1cbis})
\end{itemize}


\begin{exe}
\ex \label{ex:1cbis}
\gll
	\ipa{tʂaɕi-w}$_i$ \ipa{ŋa-gi}	\ipa{jə-rə-ge} [\ipapl{ədə}$_i$	\ipapl{dʐʚma} 	\ipapl{de} 	\ipapl{nə-se-w}] \ipa{jə-rə}  \\
	{Bkrashis-\erg} {1\sg-\dat} {say-\const-\conv} {\refl} Sgrolma {\dem} {\prf-kill-1\sg>3} say-\const\\ 
	\glt Bkrashis$_i$ told me that he$_i$ had killed Sgrolma.
\end{exe}


\begin{itemize}
\item third and second person pronouns referring to the \textsc{cs} are shifted to first person pronouns (cf. \ref{ex:1d})
\end{itemize}

\begin{exe}
\ex \label{ex:1d}
\gll
	\ipa{tʂaɕi-w}	\ipa{jə-rə-ge} [\ipapl{ŋa-w} \ipapl{dʐʚma} \ipapl{de} \ipapl{nə-f-se-sə}] \ipa{jə-rə}  \\
	{Bkrashis-\erg} {say-\const-\conv} {1\sg-\erg} Sgrolma {\dem} {\prf-\inv-kill-\evid} say-\const\\ 
	\glt Bkrashis said that I had killed Sgrolma.
\end{exe}

\begin{itemize}

\item Crucially, however, the verb does not shift to the \textsc{cs}'s perspective but retains the \textsc{os}'s one (cf. \ref{ex:1c} through \ref{ex:1d}).

\end{itemize}


% 
% \subsection{Semi-direct speech}
% 
% \begin{itemize}
%  \item \textsc{sds} in Rtau is of type II in \cite{aikhenvald08semidirect}'s terms, i.e. the current speaker (\textsc{cs}) reports an original speaker (\textsc{os})'s words (or thoughts) by partly adjusting them to their (\textsc{cs}) perspective. 
% 
% \begin{itemize}
%  \item and finally, third person pronouns referring to the \textsc{cs}'s addressee are shifted to second person pronouns (cf. \ref{ex:1e}).
% \end{itemize}
% 
% \begin{exe}
% \ex \label{ex:1e}
% \gll
% 	\ipa{tʂaɕi-w}	\ipa{jə-rə-ge} [\ipapl{ɲi} \ipapl{dʐʚma} \ipapl{de} \ipapl{nə-se-j-sə}] \ipa{jə-rə}  \\
% 	{Bkrashis-\erg} {say-\const-\conv} {2\sg} Sgrolma {\dem} {\prf-kill-2\sg>3-\evid} say-\const\\ 
% 	\glt Bkrashis said that you had killed Sgrolma.
% \end{exe}
% 
% \item Crucially, however, the verb does not shift to the \textsc{cs}'s perspective but retains the \textsc{os}'s one (cf. \ref{ex:1c} through \ref{ex:1e}).
% 
% \end{itemize}
% 

%[shrink=10]


\begin{itemize}

\item Furthermore, \textsc{sds} is used not only with verbs of reporting, but also with some verbs denoting cognitive activities such as \ipa{ntsʰə} `think' (cf. \ref{ex:2a} through \ref{ex:3b}).
\end{itemize}

\begin{exe}
\ex
\begin{xlist}
\ex \label{ex:2a}
\gll
	\ipa{ŋa} \ipa{tə-se-ã} / \ipa{tə} \ipa{tə-se}\\
	{1\sg}  {\prf-die-1} / {3\sg}  {\prf-die}\\ 
	\glt I am dead. / He is dead.

\ex \label{ex:2b}
\gll
	\ipa{ŋa} [\ipapl{tʂaɕi} \ipapl{tə-se}] \ipa{ntsʰə-w-rə}  \\
	{1\sg} {Bkrashis} {\prf-die} {think-1\sg>3-\const}\\ 
	\glt I think Bkrashis is dead.

\ex \label{ex:2c}
\gll
	\ipa{tʂaɕi-w} [\ipapl{ŋa} \ipapl{tə-se}] \ipa{ntsʰə-rə}  \\
	{Bkrashis-\erg} {1\sg} {\prf-die} {think-\const}\\ 
	\glt Bkrashis thinks I am dead.

% \ex \label{ex:3c}
% \gll
% 	\ipa{tʂaɕi-w} \ipa{ŋa}	\ipa{tə-f-se} \ipa{ntsʰə-rə}  \\
% 	{Bkrashis-\erg} {1\sg} {\prf-\inv-die} {think-\const}\\ 
% 	\glt Bkrashis thinks he killed me.

\end{xlist}
\end{exe}


\begin{itemize}

\item Furthermore, \textsc{sds} is used not only with verbs of reporting, but also with some verbs denoting cognitive activities such as \ipa{ntsʰə} `think' (cf. \ref{ex:2a} through \ref{ex:3b}).
\end{itemize}

\begin{exe}
\ex
\begin{xlist}
\ex \label{ex:3a}
\gll
	\ipa{ŋa} [\ipapl{tə-w}	 \ipapl{jɞ} \ipapl{tə-spɞ}] \ipa{ntsʰə-w-rə}  \\
	 {1\sg} {3\sg-\erg} house {\prf-move} {think-1\sg>3-\const}\\ 
	\glt I think he has moved.

\ex \label{ex:3b}
\gll
	 \ipa{tə-w} [\ipapl{ŋa-w} \ipapl{jɞ} \ipapl{tə-spɞ}] \ipa{ntsʰə-rə}  \\
	  {3\sg-\erg} {1\sg-\erg} house {\prf-move} {think-\const}\\ 
	\glt He thinks I have moved.
\end{xlist}
\end{exe}





\begin{itemize}
 \item  The most striking feature of \textsc{sds} in Rtau is that it is the only construction in the language in which the first person singular pronoun is ergative case marked when the agent of a transitive verb (cf. \ref{ex:2cbis} against \ref{ex:3bbis}).
\end{itemize}

\begin{exe}

\ex \label{ex:2cbis}
\gll
	\ipa{tʂaɕi-w} [\ipapl{ŋa} \ipapl{tə-se}] \ipa{ntsʰə-rə}  \\
	{Bkrashis-\erg} {1\sg} {\prf-die} {think-\const}\\ 
	\glt Bkrashis thinks I am dead.

\ex \label{ex:3bbis}
\gll
	 \ipa{tə-w} [\ipapl{ŋa-w} \ipapl{jɞ} \ipapl{tə-spɞ}] \ipa{ntsʰə-rə}  \\
	  {3\sg-\erg} {1\sg-\erg} house {\prf-move} {think-\const}\\ 
	\glt He thinks I have moved.
\end{exe}




 \subsection{Summary}

\begin{itemize}
 \item \textsc{sds} seems to be of the obligatory subtype in \cite{aikhenvald08semidirect}'s terms whenever the \textsc{cs} is the first person and the addressee or the object within the speech report is coreferent either with them (i.e., the \textsc{cs}) or with their (i.e., the \textsc{cs}'s) addressee. It is not a stylistic device and seems reasonably frequent in the appropriate speech or thought reporting contexts.
\end{itemize}



\section{Semi-direct  speech in Japhug} 


\subsection{The Japhug language}


\begin{itemize}
 \item  Japhug (locally known as \ipa{kɯrɯ skɤt}) is a Rgyalrongic language spoken in Mbarkham county (\zh{马尔康} Mǎ'ěrkāng xiàn) 

\includegraphics[width=0.7\textwidth]{carte.JPG}

\end{itemize}


 
 \subsection{Pronouns}

   \begin{exe}
\ex \label{ex:nWGi.kAsWso}
\gll 
\ipa{nɤ-wa}  	\ipa{kɯ}  	{``\ipa{nɤʑo}} 	\ipa{nɯɣi}"  	\ipa{kɤ-sɯso}  	\ipa{kɯ}  	\ipa{kʰa}  	\ipa{ɯ-rkɯ}  	\ipa{ʁmaʁ}  	\ipa{χsɯ-tɤxɯr}  	\ipa{pa-sɯ-lɤt}  	\ipa{ɕti}  	\ipa{tɕe}  \\
{2\sg.\poss-father} {\erg} {2\sg} {come.back:\textsc{fact}}  \infin-think \textsc{erg} house \textsc{3sg.poss}-side soldier three-circle \textsc{pfv:3$\rightarrow$3'-caus}-throw be.\textsc{affirmative}:\textsc{fact} \textsc{lnk}\\
\glt \textbf{Direct}: Your father, thinking ``{He} {is coming back}",   put three circles of soldiers around the house. 
\glt  \textbf{Indirect}: Your father, thinking that {you} are coming back,
\glt  \textbf{Semi-Direct}: Your father, thinking that {you} {is coming back}, 
\end{exe}
   
The verb \ipa{sɯso} `think' is transitive, so the absence of ergative marking on \ipa{ɯʑo} `he' shows that this pronoun is the S of the complement clause, not the A of the main verb.
      
\begin{exe}
\ex
\gll  ``{\ipa{ɯʑo}}  	\ipa{χsɯ-sŋi}  	\ipa{χsɤ-rʑaʁ}  	\ipa{ma}  	{\ipa{mɯ-pɯ-rɤʑi-a}}"  	\ipa{ɲɯ-nɯ-sɯsɤm}  	\ipa{pjɤ-ŋu}  \\
{\textsc{3sg}} three-day  three-night apart.from {\textsc{neg-pst.ipfv}-stay-\textsc{1sg}} \textsc{ipfv-auto}-think[III] \textsc{evd.ipfv}-be \\
\glt    \textbf{Direct}: He was thinking ``{I} {have} only {stayed} for three days and three nights."
\glt    \textbf{Indirect}: He was thinking that {he} had only stayed for three days and three nights.
\glt  \textbf{Semi-Direct}: He was thinking that {he} {have} only {stayed} for three days and three nights. 
\end{exe}
  
 
 
  
\begin{exe}
\ex
\gll \ipa{tɕe}  	``{\ipa{tɕhi}}  	{\ipa{pɯ-sat-a}}"  	\ipa{nɯ}  	\ipa{mɯ-ko-rɤt}  	\ipa{kɯ,}  	``\ipa{tɯtɯrca}  	\ipa{kɯɕnɯz}  	\ipa{pɯ-sat-a.}"  	\ipa{ko-rɤt.}  \\
\textsc{lnk} {what} {\textsc{pfv}-kill-\textsc{1sg}}  \textsc{dem} \textsc{neg-evd}-write \textsc{erg} together seven \textsc{pfv}-kill-\textsc{1sg} \textsc{neg-evd}-write \\
\glt Direct: He did not write {what} he had killed,  he (just) wrote ``I killed seven (of them)."
 \end{exe}
 
 
    
   \subsection{Possessive prefixes}
\begin{exe}
\ex
\gll  \ipa{tɕe}  	\ipa{ta-ʁi}  	\ipa{nɯ}  	\ipa{kɯ}  	``{\ipa{ɯ-pi}}  	\ipa{ɣɯ}  	\ipa{ɯ-sci}  	{\ipa{tu-nɤme-a}}  	\ipa{ra}" 	\ipa{ɲɤ-sɯso}  	\ipa{tɕe,}  	\\
\textsc{lnk}  \textsc{indef.poss}-younger.sibling \textsc{dem} \textsc{erg}  {\textsc{3sg.poss}-elder.sibling}  \textsc{gen} \textsc{3sg.poss}-revenge {\textsc{ipfv}-make[III]-\textsc{1sg}} have.to:\textsc{fact} \textsc{evd}-think \textsc{lnk} \\
\glt  \textbf{Direct}: The (younger) sister thought ``{I have to get revenge} on {my brother}".
\glt  \textbf{Indirect}:  The (younger) sister wanted to get revenge on {her brother}".
\glt  \textbf{Semi-direct}:  The (younger) sister$_i$ thought {I have to get revenge} on {her$_i$ brother}".
  \end{exe}
  
 
\begin{exe}
\ex
\gll   \ipa{tɤɕime}  	\ipa{nɯ}  	\ipa{kɯ}  	\ipa{pjɯ-tɯ-mtshɤm}  	\ipa{tɕe,}  	\ipa{nɯnɯ}  {\ipa{ɯ-kɯmtɕhɯ}}  	\ipa{nɯ}  	{\ipa{ju-ɣɯt-a}}  	\ipa{ŋu}  		\ipa{ɯ-kɯ-ti}  	\ipa{pjɤ-tu}  	\ipa{ndɤre,}  \\
girl \textsc{dem} \textsc{erg} \textsc{ipfv-conv:imm}-hear \textsc{lnk} \textsc{dem} {\textsc{3sg.poss}-toy} \textsc{dem} {\textsc{ipfv}-bring-\textsc{1sg}}  be:\textsc{fact} \textsc{3sg-nmlz}:S/A-say \textsc{evd.ipfv}-exist \textsc{lnk} \\
\glt   \textbf{Direct}: As soon as the girl heard that there was someone saying ``{I will bring} {your toy}".
\glt   \textbf{Indirect}:  As soon as the girl heard that there was someone saying that he would bring {her toy}.
\glt   \textbf{Semi-direct}: As soon as the girl$_i$ heard that there was someone saying ``{I will bring} {her$_i$ toy}".

  \end{exe}
  

 
%       
%         \subsection{Possessive prefixes}
% \begin{exe}
% \ex
% \gll   {\ipa{a-tʂɯnlɤn}}  	{\ipa{ɲɯ-nɯ-fsɯɣ-a}}  	\ipa{ɯ-ɲɯ-tɯ-sɯsɤm}  	\ipa{nɤ,}  	\ipa{nɯ}  	\ipa{tɤ-ste}  	\ipa{ti}  \ipa{ɲɯ-ŋu} \\
%  {\textsc{1sg.poss}-favour} {\textsc{ipfv-auto}-pay.back-\textsc{1sg}} \textsc{cond-ipfv}-2-think[III] \textsc{lnk} \textsc{dem} \textsc{imp}-do.this.way[III] say:\textsc{fact} \textsc{testim}-be \\
%  \glt    \textbf{Direct}: If you think ``{I will requite} the {favour} (which I received from {you})", do like that.
% \glt    \textbf{Indirect}: If you want to requite the {favour} (which you received from {me}), do like that.
%   \end{exe}
%   

        \bibliographystyle{linquiry2.bst}
	\def\newblock{}        
\bibliography{../biblioswl}


\end{document}


