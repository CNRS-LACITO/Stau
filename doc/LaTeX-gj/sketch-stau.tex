\documentclass[oneside,a4paper,11pt]{article} 
\usepackage{fontspec}
\usepackage{natbib}
\usepackage{booktabs}
\usepackage{xltxtra} 
\usepackage{longtable}
\usepackage{polyglossia}
\usepackage[top=72pt,bottom=72pt,left=72pt,right=72pt]{geometry}
\usepackage[table]{xcolor}
\usepackage{gb4e} 
\usepackage{multicol,multirow}
\usepackage{graphicx}
\usepackage{float}
\usepackage{slashbox} 
\usepackage{rotating}
\usepackage{hyperref} 
\hypersetup{bookmarksnumbered,bookmarksopenlevel=5,bookmarksdepth=5,colorlinks=true,linkcolor=blue,citecolor=blue}
\usepackage[all]{hypcap}
\usepackage{memhfixc}
\usepackage{lscape}
\bibpunct[: ]{(}{)}{,}{a}{}{,}
%%%%%%%%%quelques options de style%%%%%%%%
%\setsecheadstyle{\SingleSpacing\LARGE\scshape\raggedright\MakeLowercase}
%\setsubsecheadstyle{\SingleSpacing\Large\itshape\raggedright}
%\setsubsubsecheadstyle{\SingleSpacing\itshape\raggedright}
%\chapterstyle{veelo}
%\setsecnumdepth{subsubsection}
%%%%%%%%%%%%%%%%%%%%%%%%%%%%%%%
\setmainfont[Mapping=tex-text,Numbers=OldStyle,Ligatures=Common]{Charis SIL} %ici on définit la police par défaut du texte
\renewcommand \thesection {\arabic{section}.}
\renewcommand \thesubsection {\arabic{section}.\arabic{subsection}.}
\newfontfamily\phon[Mapping=tex-text,Ligatures=Common,Scale=MatchLowercase,FakeSlant=0.3]{Charis SIL} 
\newcommand{\ipa}[1]{{\phon #1}} %API tjs en italique
 
\newcommand{\grise}[1]{\cellcolor{lightgray}\textbf{#1}}
\newfontfamily\cn[Mapping=tex-text,Scale=MatchUppercase]{MingLiU}%pour le chinois
\newcommand{\zh}[1]{{\cn #1}}

\newcommand{\jg}[1]{\ipa{#1}\index{Japhug #1}}
\newcommand{\wav}[1]{#1.wav}
\newcommand{\tgz}[1]{\mo{#1} \tg{#1}}

\XeTeXlinebreaklocale 'zh' %使用中文换行
\XeTeXlinebreakskip = 0pt plus 1pt %
 \newcommand{\ra}{$\Sigma_1$} 
\newcommand{\rc}{$\Sigma_3$} 
\newcommand{\ro}{$\Sigma$} 

 
 
\begin{document} 


\title{Stau Horpa}
\author{Guillaume Jacques, Anton Antonov, Lai Yunfan and Lobsang Nima}

\maketitle

 

\section{Introduction}
The cluster of languages variously referred to as Stau, Ergong or Horpa in the literature are spoken over a large area from Ndzamthang county (in Chinese Rangtang  \zh{壤塘县})  in Rngaba  prefecture (Aba \zh{阿坝州}) to Rtau county (Daofu \zh{道孚}) in Dkarmdzes prefecture (Ganzi \zh{甘孜州}), in sichuan province, China. At the moment of writing, it is still unclear how many unintelligible varieties belong to this group, but at least three must be distinguished: the language of Rtau county (referred as `Stau' in this paper), the Dgebshes language (Geshizha \zh{格什扎话})  spoken in Rongbrag county (Danba \zh{丹巴}), and the Stodsde language (Shangzhai \zh{上寨}) in Ndzamthang.  

Research on these languages is still limited. No dictionaries or text corpora have yet been published. Stodsde is only known by a few articles (\citet{jackson00sidaba} and   \citet{jackson07shangzhai}), Dgebshes by a grammatical sketch (\citealt{duoerji98geshizha}). The dialects spoken in Rtau county have been investigated by several teams of scholars (\citet{huangbf91daofu}, \citet{sun13gexi} and \citealt{jacques14rtau}), but no detailed description of this language is available.

There is no consensus as to how these languages should be named. The present paper favours the Tibetan names of these languages rather than the Chinese-based ones, since Chinese names are transcription of the Tibetan names. 

  The native speaker among the authors (Lobsang Nima) favours the name \ipa{rəsɲəske} for his language (the Khang.gsar \ipa{qʰərŋe} dialect in Rtau county, in Chinese \zh{孔色} Kongse), but this name is not used by all speakers and we prefer the geographically-based name  `Stau'. The spelling with \textit{St--} rather than \textit{Rt--}, aside from being more pronounceable for the average western reader, better reflects the local pronunciation of the county name \ipa{stʚwu}. This spelling has already been used in English (see \citealt{wang70stau}).   
  
From the group comprising all three languages (Stau, Dgebshes and Stodsde),  we can follow 
\citet{jackson00sidaba}'s term `Horpa' which has indeed been used for this area in the past, or the related `Tre-Hor'.  The term Ergong \zh{尔龚} used by scholars such as \citet{sun83liujiang}, on the other hand, appears to lack any basis in the local languages and should be avoided.
 
As shown in section \ref{sec:classification}, there is evidence from verbal morphology and lexicon that Horpa languages form a subgroup within Rgyalrongic with Khroskyabs (previously known by the incorrect term `Lavrung'), as these two branches present common innovations that are unlikely to be explainable as parallel developments.
 
\section{Phonology}
 
 Voice stops:
 \ipa{bəlʚ} `cheek' vs \ipa{pərje} 
 
 \subsection{Vowel fusion}
 
  \citealt{jacques14rtau}
\begin{table}[H]
\caption{Vowel fusion in Stau} \label{tab:alternation} \centering
\begin{tabular}{c|cccccc}
\toprule

 \backslashbox{Stem}{Suffix} &  	\textsc{1sg$\rightarrow$3} --\ipa{w} & 1 --\ipa{ã} & 2$\rightarrow$3 --\ipa{j} \\
\hline
\ipa{i}&\ipa{u}&\ipa{ã}&\ipa{i}\\
\ipa{e}&\ipa{ow}&\ipa{ã}&\ipa{ej}\\
\ipa{a}&\ipa{ow}&\ipa{ã}&\ipa{ej}\\
\ipa{ə}&\ipa{u}&\ipa{õ}&\ipa{i}\\
\ipa{ʚ}&\ipa{ow}&\ipa{õ}&\ipa{ej}\\
\bottomrule
\end{tabular}
\end{table}

 
\section{Verbal morphology}




\subsection{Intransitive conjugation}
In the intransitive conjugation, the third and second persons singular forms are in the bare stem, while first person (singular and plural) forms have a suffix \ipa{--ã}.

Following the rules of  vowel fusion in Table \ref{tab:alternation}, vowel fusion between the first person suffix \ipa{--ã} and the verb stem has differ results depending on the vowel (and coda) of the verb stem. 
 
Six classes of alternations are found in verbs with open syllables; class 6 includes verbs without alternation, whose rhyme can be any of --\ipa{u}, --\ipa{oˠ}, --\ipa{aˠ}, --\ipa{õ} and --\ipa{ã}:
\begin{table}[H]
\caption{Vowel alternations in open-syllable intransitive verbs in Rtau} \label{tab:open.intr} \centering
\begin{tabular}{llll|ll|l}
\toprule
&1&2&3&4&5&6 \\
meaning &	look at   &  	move   &  	like&  	be full     &  	 	be ill      &  	be hot       \\  
\midrule
1&	\ipa{scəqã} & 	\ipa{mbəɕã} & \ipa{rgã} &	\ipa{fkõ} & 	  	\ipa{ŋõ} & 	   	\ipa{cʰu}   \\ 
1 (underlying)&	\ipa{scəqi-ã} & 	\ipa{mbəɕe-ã} & \ipa{rga-ã} &	\ipa{fkə-ã} & 	  	\ipa{ŋɞ-ã} & 	   	\ipa{cʰu-ã}   \\ 
2/3&	\ipa{scəqi} & 	\ipa{mbəɕe} & \ipa{rga} & 	\ipa{fkə} & 	  	\ipa{ŋɞ} & 	 	\ipa{cʰu}  \\ 
\bottomrule
\end{tabular}
\end{table}

%The alternations can be stated in a straightforward way: centralized vowels  --\ipa{ə} and  --\ipa{ɞ} change to -\ipa{õ}, and front and open (unrounded and non-velarized) vowels change to --\ipa{ã}.
%
%In the case of verb with stems ending in--\ipa{r} or --\ipa{v}, the first person is always derived by the replacing the entire rhyme by --\ipa{ã} or \ipa{õ} depending on the main vowel of the rhyme:
%\begin{table}[H]
%\caption{Vowel alternations in closed syllable intransitive verbs in Rtau} \label{tab:close.intr} \centering
%\begin{tabular}{llll|ll|l}
%\toprule
%meaning &	sleep   &  	hide   \\  
%\midrule
%1&	\ipa{ɲɟã} & 	\ipa{ɲcʰã} \\ 
%2/3&	\ipa{ɲɟev} & 	\ipa{ɲcʰer} \\ 
%\bottomrule
%\end{tabular}
%\end{table}
%
%Stems ending in --\ipa{m} (the only other final consonant available) are always Tibetan loanwords and do not present any alternation.
%
%
%There are two irregular verbs with intransitive morphology in Japhug with have \ipa{--ã} in the first person instead of expected \ipa{--õ} (cf table \ref{tab:irr.intr}). 
%
%\begin{table}[H]
%\caption{Irregular intransitive verbs in Rtau} \label{tab:irr.intr} \centering
%\begin{tabular}{llll|ll|l}
%\toprule
%meaning &	go     & say \\  
%\midrule
%1&	\ipa{ɕã}  	 &\ipa{jã}\\ 
%2/3&	\ipa{ɕə} & 	\ipa{jə} &\\ 
%\bottomrule
%\end{tabular}
%\end{table}
%
% 
%
%If we disregard the irregular verbs, it is always possible to determine the first person from the second/third person form. Thus, we may posit that the 2/3 person represents the bare stem, and that the first person is derived from it by fusion with a suffix --\ipa{ã}, which is realized as --\ipa{õ} when the rhyme is centralized.
%
 
%
%
%The transitive conjugation includes at most six different forms, illustrated by the paradigms in tables \ref{tab:kill} and \ref{tab:give} (these paradigms are in the perfective form with directional prefixes, which can be neglected for the purpose of the present paper). Some of these forms are distinguished by the presence of a prefix \ipa{f}-- / \ipa{v}-- whose nature is analysed in more detail in section \ref{sec:alignment} If we disregard this prefix, only four different stems at most are distinguished: \textsc{1sg$\rightarrow$3}, 2$\rightarrow$3, \textsc{1pl$\rightarrow$3} (which has the same vocalism as 2/3$\rightarrow$1) and third person (which has the same vocalism as 1$\rightarrow$2).
%
%
%
%\begin{table}[h]
%\caption{\ipa{fse} `kill'}
%\centering \label{tab:kill}
%\begin{tabular}{|c|cc|c|c|}  
% \cline{1-5}
%\backslashbox{A}{P} &1s  &  1p  &  2  &  	3  \\  
%\cline{1-5}
% 1s  &  	 \multicolumn{2}{c}{\cellcolor{lightgray}}   \vline    &  	\multirow{2}{*}{\ipa{nə-se}}  &  	\ipa{nə-sow}  \\  
%\cline{5-5}1p  &  \multicolumn{2}{c}{\cellcolor{lightgray}} 	 \vline   &   &  	\ipa{nə-sã}  \\  
%\cline{5-5}2 &    \multicolumn{2}{c}{\multirow{2}{*}{\ipa{nə-fsã}}}    \vline  &   \grise{ }	  &  	\ipa{nə-sej}  \\  
%\cline{5-5}3 &  \multicolumn{2}{c}{ } \vline &  	\multicolumn{2}{c}{ \ipa{nə-fse}}   	 \vline  \\  
%\cline{1-5}
%\end{tabular}
%\end{table}
%
%\begin{table}[h]
%
%\caption{\ipa{f-kʰʚ} `give'}
%\centering \label{tab:give}
%\begin{tabular}{|c|c|c|c|c|}  
% \cline{1-4}
%\backslashbox{A}{R}  &  	1   &  	2  &  	3  \\  
%\cline{1-4}
% 1s  &   \cellcolor{lightgray}       &  	\multirow{2}{*}{\ipa{tə-kʰʚ}}  &  	\ipa{tə-kʰow}  \\  
%\cline{4-4}1p  &   \cellcolor{lightgray}	    &  & \ipa{tə-kʰõ}  \\  
%\cline{2-4}2 &     \multirow{2}{*}{\ipa{tə-fkʰõ}}    &   \grise{ } &  	\ipa{tə-kʰe}\\  
%\cline{3-4}3 &     & \multicolumn{2}{c}{ \ipa{tə-fkʰʚ} } \vline \\  
%\cline{1-4}
%\end{tabular}
%\end{table}
%
%
%As with intransitive verbs, six classes of verb alternation are attested in transitive conjugations, depending on the final vowel of the verb stem. Table \ref{tab:open.tr} presents all six classes (it contains the verbs stems without the inverse prefix \ipa{f}--/\ipa{v}--.) Class 6  includes all verbs with stem ending in --\ipa{u}, --\ipa{oˠ}, --\ipa{aˠ}, --\ipa{õ} and --\ipa{ã}.
%
%\begin{table}[H]
%\caption{Vowel alternations in open-syllable transitive verbs in Rtau} \label{tab:open.tr} \centering
%\begin{tabular}{llll|ll|ll}
%\toprule
%&1&2&3&4&5&6 \\
%meaning &	drink & kill & dig & dress up & give &cut
%\\
%\midrule
%\textsc{1sg$\rightarrow$3}&	--\ipa{tʰu}&--\ipa{sow}&--\ipa{ɴqʰʚrow}&--\ipa{zgu}&--\ipa{kʰow}&--\ipa{tsu}&
%\\
%\textsc{1pl$\rightarrow$3}, 2/3$\rightarrow$1& --\ipa{tʰã}&--\ipa{sã}&--\ipa{ɴqʰʚrã}&--\ipa{zgõ}&--\ipa{kʰõ}&--\ipa{tsu}&
%\\
%2$\rightarrow$3& --\ipa{tʰi}&--\ipa{sej}&--\ipa{ɴqʰʚrej}&--\ipa{zgi}&--\ipa{kʰej}&--\ipa{tsu}&
%\\
%3$\rightarrow$3, 1$\rightarrow$2&--\ipa{tʰi}&--\ipa{se}&--\ipa{ɴqʰʚra}&--\ipa{zgə}&--\ipa{kʰʚ}&--\ipa{tsu}&
%\\
%\bottomrule
%\end{tabular}
%\end{table}
%As with intransitive verbs, it is possible to regard the third person form as the basic one; the \textsc{1pl$\rightarrow$3} and 2/3$\rightarrow$1 stems can be analysed as resulting from fusion with the first person --\ipa{ã} suffix. 
%The \textsc{1sg$\rightarrow$3} form presents rounding of the vowels with an additional \ipa{-w} glide in the case of mid-low and low vowels. These alternations can be accounted for by assuming the existence of a suffix whose underlying form is \ipa{--w}.
%
%The \textsc{2$\rightarrow$3} form has vowel fronting with an additional \ipa{-j} glide for  mid-low and low vowels. Here the underlying form \ipa{--j} can be posited.
%
%In closed syllables, final consonants differ as to their behaviour with the person suffixes. Final \ipa{--v} drops with the \textsc{1sg$\rightarrow$3} --\ipa{w} and first person --\ipa{ã} suffixes; the second person \ipa{--j} suffix does not cause final --\ipa{v} to drop but nevertheless induces vowel fronting as in --\ipa{zgriv} `you accomplished'. Final --\ipa{m} is immune to any change from the suffixes and verbs ending in this consonant present no stem alternations. Final --\ipa{r} drops with all three suffixes --\ipa{w}, --\ipa{ã} and --\ipa{j} and the final consonant is preserved on the in the third person and 1$\rightarrow$2 forms.
%
%\begin{table}[H]
%\caption{Vowel alternations in closed syllable transitive verbs in Rtau} \label{tab:close.tr} \centering
%\begin{tabular}{lll|l|ll|ll}
%\toprule
%meaning &accomplish& give back& close&rob
%\\
%\midrule
%\textsc{1sg$\rightarrow$3}&	--\ipa{zgru}&--\ipa{xsow}&--\ipa{zdəm}&--\ipa{stow}
%\\
%\textsc{1pl$\rightarrow$3}, 2/3$\rightarrow$1& --\ipa{zgrõ}&--\ipa{xsõ}&--\ipa{zdəm}&--\ipa{stõ}
%\\
%2$\rightarrow$3& --\ipa{zgriv}&--\ipa{xsev}&--\ipa{zdəm}&--\ipa{stej}
%\\
%3$\rightarrow$3,1$\rightarrow$2&--\ipa{zgrəv}&--\ipa{xsev}&--\ipa{zdəm}&--\ipa{stʚr}
%\\
%\bottomrule
%\end{tabular}
%\end{table}
%
% 
% 
% 
%
%
 
%
%With the morphophonological rules presented in the previous section, it is possible to present the Rtau paradigms in condensed format as in Table \ref{tab:align}.
%
%\begin{table}[h]
%\caption{Rtau transitive and intransitive paradigms}
%\centering \label{tab:align}
%\begin{tabular}{|c|c|c|c|c|}  
% \cline{1-4}
%\backslashbox{A}{P} &1    &  2  &  	3  \\  
%\cline{1-4} 1s  &   \cellcolor{lightgray}        &  	\multirow{2}{*}{\ro{}}  &  	\ro{}-\ipa{w}  \\  
%\cline{4-4}1p  &   \cellcolor{lightgray} 	     &   &  	\ro{}-\ipa{ã}  \\  
%%\hline
%\cline{4-4}2 &   \multirow{2}{*}{\ipa{v}-\ro{}-\ipa{ã}}     &   \grise{ }	  &  	\ro{}-\ipa{j}  \\  
%\cline{3-4}3 &    &  	\multicolumn{2}{c}{ \ipa{v}-\ro{}}   	 \vline  \\  
%\hline
%\textsc{intr}&\ro{}-\ipa{ã}  &\multicolumn{2}{c}{  \ro{}}     	 \vline  \\  
%\hline
%\end{tabular}
%\end{table}
%
%The suffixes \ipa{--w} and \ipa{--j} are restricted to transitive \textit{direct} forms (with an SAP agent and third person patient); they are not found in intransitive and \textit{inverse} forms (with an SAP patient and third person agent). The first person default suffix \ipa{--ã} appears in all forms involving the first person except \textsc{1sg$\rightarrow$3} (where the suffixal slot is occupied by \ipa{--w}) and 1$\rightarrow$2. 
%
%The absence of the suffix  \ipa{--ã}  in 1$\rightarrow$2 is not surprising. In all Rgyalrongic languages, as well as in neighbouring languages such as Tangut (see for instance \citealt[18]{jacques09tangutverb}, \citealt{gongxun14agreement}, \citealt{lai14person}), in local 1$\rightarrow$2 and 2$\rightarrow$1 forms suffixes are coreferent with the P (except in the case of double suffixation). Since the second person S/P suffix is zero, the absence of any suffix in the 1$\rightarrow$2 form is expected.
%
%
 
%The \ipa{f}-- / \ipa{v}-- prefix appears in 2/3$\rightarrow$1, 3$\rightarrow$2 and 3$\rightarrow$3 forms. Its presence  in 2$\rightarrow$1 precludes an analysis as a third person agent marker, and it is best to treat it as an inverse marker.
%
%
%
%The inverse \ipa{v}-- prefix appears in all 3$\rightarrow$3 forms in Rtau, a feature shared with Khroskyabs. Both Rtau and Khroskyabs differ from Rgyalrong languages, where two 3$\rightarrow$3 forms are found: the \textit{direct}   and the \textit{inverse} form. 
%The inverse \ipa{v}-- prefix presents phonological alternations and phonotactic constraints. It is prefixed to the first syllable of the verb stem, even when polysyllabic. In verbs with reduplicated stem, such a `wipe' (Table \ref{tab:wipe}, \ipa{nə--} here is the directional prefix, see section \ref{sec:dir.pref}), reduplication also applies to the inverse prefix.
%
%  \begin{table}[H]
%\caption{\ipa{f-ɕə-f-ɕe} `wipe'}
%\centering \label{tab:wipe}
%\begin{tabular}{|c|c|c|c|c|}  
% \cline{1-4}
%\backslashbox{A}{P} &1    &  2  &  	3  \\  
%\cline{1-4} 1s  &   \cellcolor{lightgray}        &  	\multirow{2}{*}{\ipa{nə-ɕəɕe}}  &  	\ipa{nə-ɕəɕo-w}  \\  
%\cline{4-4}1p  &   \cellcolor{lightgray} 	     &   &  	\ipa{nə-ɕəɕ-ã}  \\  
%%\hline
%\cline{4-4}2 &   \multirow{2}{*}{\ipa{nə-fɕəfɕ-ã}}     &   \grise{ }	  &  	\ipa{nə-ɕəɕe-j}  \\  
%\cline{3-4}3 &    &  	\multicolumn{2}{c}{ \ipa{nə-fɕəfɕe}}   	 \vline  \\  
%\hline
%\end{tabular}
%\end{table}
%
%The \ipa{v}-- prefix is assimilated  to \ipa{f}-- when prefixed to a verb stem with unvoiced initial consonant (as in \ipa{f-se} [\textsc{inv}-kill] `he kills'). It cannot be inserted whenever any of the following three conditions apply:
%
%\begin{itemize}
%\item When the stem-initial consonant is a labial (either /\ipa{p}/, /\ipa{b}/, /\ipa{m}/, or  /\ipa{v}/) or the voiced uvular /\ipa{ʁ}/, the inverse cannot be prefixed. Thus the third person form of \ipa{və} `do' \ipa{ʁʚ} `help' are identical to the corresponding bare stems.
%\item The inverse prefix is not compatible with most stem-initial clusters. The only clusters that allow prefixation of \ipa{v}-- are /stop+r/ clusters. For instance, the root /\ipa{kʰrə}/ `hold' (\textsc{1sg$\rightarrow$3} \ipa{kʰru}) thus has a 3$\rightarrow$3 form \ipa{f-kʰrə}, whereas \ipa{zjə} `sell' has a third person form identical to the bare stem (the cluster *\ipa{vzj}-- is not allowed in the variety of Rtau under study).
%\item The inverse does not appear in transitive verbs with final \ipa{--v}, due to a dissimilatory constraint. For instance, the 3$\rightarrow$3 forms of /\ipa{kʰev}/ `scoop' and /\ipa{ɕev}/ `take out' are \ipa{\ipa{kʰev}} and \ipa{\ipa{ɕev}} respectively, not *\ipa{fkʰev} or *\ipa{fɕev}.
%\end{itemize}
%
 
%
%The morphologically based distinction between transitive and intransitive verbs in Rtau must be refined by taking into account case-marking on arguments. 
%
%Rtau, as all Rgyalrongic languages, is a strict verb-final language with postpositions. Case markers include the ergative \ipa{--w}, the genitive \ipa{--j}, the dative \ipa{--gi} and the instrumental \ipa{--kʰa}. Only animate referents can receive   ergative marking, inanimates can only be marked with the instrumental. SAP pronouns are not normally marked with the ergative (except in some subordinate clauses).
%
%
%Some verbs with intransitive morphology, such as `like', do require ergative marking on the argument whose person is indexed on the verb, as illustrated by examples \ref{ex:rgaN} and \ref{ex:rga}.
%\begin{exe}
%\ex \label{ex:rgaN}
%\gll \ipa{ŋa}  	\ipa{tə-ɡi}  	\ipa{rɡa-ã-rə}  \\
%I he-\textsc{dat} like-1-\textsc{const} \\
%\glt `I like him/her.'
%\end{exe}
%
%\begin{exe}
%\ex \label{ex:rga}
%\gll \ipa{tə-w}  	\ipa{ŋa-ɡi}  	\ipa{rɡa-rə}  
% \\
%he-\textsc{erg} I-\textsc{dat} like-\textsc{const} \\
%\glt `(S)he likes me.'
%\end{exe}
%
%This type of semi-transitive verb (which are transitive from the point of view of case marking and intransitive from that of verb morphology) include some experiencer verbs like \ipa{rga} `like' and some speech verbs like \ipa{jə} `say'. The stimulus or the addressee is marked with the dative case.
%
%Some verbs with transitive morphology agree with only one of their arguments. Thus, /si/ `know (somebody)' indexes  the person knowing, while the P is always third person by default, as shown in Table \ref{tab:know}.
%
%\begin{table}[H]
%\caption{\ipa{f-si} `know'}
%\centering \label{tab:know}
%\begin{tabular}{|c|c|c|c|c|}  
% \cline{1-4}
%\backslashbox{A}{P} &1    &  2  &  	3  \\  
%\cline{1-4} 1s  &   \cellcolor{lightgray}        &  	\multicolumn{2}{c}{\ipa{su}}  \vline  \\  
%\cline{3-4}1p  &   \cellcolor{lightgray} 	     &  \multicolumn{2}{c}{\ipa{sã}}\vline  \\  
%%\hline
%\cline{4-4}2 &    \ipa{si}     &   \grise{ }	  &  	 \ipa{si}  \\  
%\cline{2-4}3 &     	\multicolumn{3}{c}{ \ipa{fsi}}   	 \vline  \\  
%\hline
%\end{tabular}
%\end{table}
%
%When the person known is an SAP, an overt  pronoun must be used, and appears in the absolutive form (example \ref{ex:fsi}).
%
%\begin{exe}
%\ex \label{ex:fsi}
%\gll \ipa{tə-w}  	\ipa{ŋa}  	\ipa{f-si}  
% \\
%he-\textsc{erg} I \textsc{inv}-know \\
%\glt `S/he knows me'.
%\end{exe}
%
%
%Ditransitive verbs that index the recipient as the P (\textit{secundative} in \citealt{malchukov10ditransitive}'s terminology), the recipient still receives dative marking despites being indexed in the verb morphology, as in example \ref{ex:xsev} with the verb /xsev/ `give back'.
%
%\begin{exe}
%\ex \label{ex:xsev}
%\glt \ipa{təɲu ŋaɲəgi kəxsã}
%\gll
%\ipa{tə-ɲə-w}  	\ipa{ŋa-ɲə-gi}  	\ipa{kə-v-xsev-ã.}  \\
%3-\textsc{pl-erg} 1-\textsc{pl-dat} \textsc{pfv-inv}-return-1 \\
%\glt They gave it back to us.
%\end{exe}
%
 
%As in all Rgyalrongic languages, Rtau has a system of five directional prefixes used to indicate both direction and TAM. The prefixes come in two series, one used for perfective and imperative forms (with \ipa{ə} vocalism), and another one for perfective interrogative (with \ipa{i} vocalism and stress), as indicated in Table \ref{tab:dir.pref}.
%
%
%
%\begin{table}[H]
%\caption{Directional prefixes in Rtau} \label{tab:dir.pref} \centering
%\begin{tabular}{lccccc}
%\toprule
%Direction & Perfective / Imperative & Interrogative \\
%\midrule
% Up & \ipa{rə--} & \ipa{rí--} \\
%Down & \ipa{nə--} & \ipa{ní--} \\
%North  & \ipa{kə--} & \ipa{kí--} \\
%South & \ipa{ɣə--} & \ipa{ɣí--} \\
%No direction & \ipa{tə--} & \ipa{tí--} \\
%\bottomrule
%\end{tabular}
%\end{table}
%
%The prefixes \ipa{kə--} and \ipa{ɣə--} are here glossed as `north' and `south' rather than `translocative' (\zh{离心}) and  `cislocative' (\zh{向心}) as in \citet[26]{huangbf91daofu}. At least in the variety under study, the  use of these two prefixes is not determined by the relative direction towards or away from the main referent. For instance, in example \ref{ex:kE},   the prefix \ipa{kə--} appears with  the verb \ipa{ʂɸa} `come out, appear' (which is compatible with all directional prefixes) to express motion towards the main referent.
%
%\begin{exe}
%\ex \label{ex:kE}
%\gll 	\ipa{tʰaˠdʑi}  	\ipa{tʰaˠdʑi-jəkʰa}  	\ipa{raca}  	\ipa{kə-ʂɸa}  	\ipa{vdʚ-sə}  	\ipa{ŋə-rə}  \\
%far far-from horseman \textsc{pfv:north}-come.out see-\textsc{evd} be-\textsc{testim} \\
%\glt (Akhu stonba) saw  a horseman coming from afar (towards him from the south to the north). (Akhustonba and the horseman, 4)
%\end{exe}
%
%Verbs other than motion verbs generally  only allow one direction, which is lexically determined. Thus for instance \ipa{tʰi} `drink' appears with \ipa{ɣə--} `towards south' with \ipa{ŋgə} `eat' is used with \ipa{nə--} `down'.
%
%Unlike Rgyalrong languages and Khroskyabs, there is no regular stem alternation in Rtau related to TAM categories. However, there are two types of irregularities in TAM marking.
%
%First, a handful of verbs are never used with directional prefixes: this is the case of \ipa{vdʚ} `see' (example \ref{ex:kE} above; the evidential form in --\ipa{sə} normally requires a directional prefix), \ipa{ste} `finish', \ipa{si} `know'.
%
%Second, the motion verbs `come' and `go' are exceptional in that they allow directional prefixes in the non-past. The presence vs absence of directional prefixes is the only difference between perfective and non-past in most verbs, but in the case of \ipa{ɮde} `come' and \ipa{ɕə} `go' the suppletive stems  \ipa{--kʰi} and \ipa{--vi} respectively are used in the non-past with directional prefixes, as summarized in Table \ref{tab:motion.irr}.\footnote{There is in addition a defective motion verb \ipa{rja} `leave' only used in the third person perfective form; for the first and second person, corresponding forms of the verb \ipa{ɕə} must be used instead.}
%
%\begin{table}[H]
%\caption{Directional prefixes in Rtau} \label{tab:motion.irr} \centering
%\begin{tabular}{lccccc}
%\toprule
%Meaning & Perfective & Non-Past & Non-Past with directional prefixes \\
%\midrule
%go & \ipa{ɕə} &\ipa{ɕə} & \ipa{--vi} \\
%come & \ipa{ɮdi} &\ipa{ɮde} & \ipa{--kʰi} \\
%\bottomrule
%\end{tabular}
%\end{table}
%
%The verb `come' has distinct perfective and non-past stems. In the perfective \ipa{ɮdi} is most often used   without directional prefix (example \ref{ex:lZdi}), but using it with directional is nevertheless possible, unlike verbs such as \ipa{vdʚ} `see'.
%
%\begin{exe}
%\ex \label{ex:lZdi}
%\gll 
%\ipa{sa}  	\ipa{ʁjikʰoˠ}  	\ipa{rʚ}  	\ipa{ɮdi-sə}  	\ipa{ŋə-rə.}   \\
%place Gyukhog up come-\textsc{evd} be-\textsc{testim} \\
%\glt He came (up there) at the place (called) Gyukhog. (The thieves, 39)
%\end{exe}
%
%The non-past stem \ipa{ɮde} occurs in the non-past and imperative forms, and it is homophonous with the transitive verb |\ipa{ɮde}| `bring' (whose 3$\rightarrow$3 form is \ipa{vɮde} with the inverse prefix, thus never ambiguous with the intransitive verb).

 \subsection{Derivational morphology}
 
 Denominal
 \ipa{pəcʰa} \ipa{sbəcʰa}
  \ipa{rmi} \ipa{smi}
    \ipa{rmi} \ipa{smi}
        \ipa{ɣme} `wound'\ipa{smi}
 
 
 Anticausative
 \ipa{səla}  \ipa{zəla}  `fall'
\ipa{pʰre}   \ipa{bre}  `break'
  \ipa{fkʰe} `cut down'  \ipa{vge} `break away, off' 
\ipa{ftʂə} \ipa{brə} `wake up' 
\ipa{ftɕə} \ipa{dʑə} `melt' 
 
 
 Causative
 \ipa{lə}  `boil' \ipa{zɮdə}
  \ipa{cʰu}  `hot' \ipa{scʰu}
    \ipa{rŋi}  `borrow' \ipa{sŋi}
    \ipa{qə}  `go out (fire)' \ipa{sqʰə}
    \ipa{tʰi}  `drink' \ipa{stʰi}
    \ipa{kə}  `wear' \ipa{zgə}
     \ipa{nə}  `burn' \ipa{snə}
     \ipa{bərje}  `burn' \ipa{spərje}

 Tibetan loanwords \ipa{mbjer} \ipa{zɟwer}
 
\ipa{ɣ} prefix
     \ipa{ndʑi} `learn'      \ipa{ɣʑi} `teach'
     

         
    applicative
    \ipa{qʰe}  `laugh' \ipa{sqʰe} `laugh at'
 zja?
 
 
     tropative
       \ipa{nənʚ}  `smell' \ipa{snəsnʚ}

incorporation

\ipa{rvatɕa} \ipa{rva}
\ipa{mbarji} `stride over' \ipa{pa}

\ipa{-r} \ipa{spar} (\ipa{ɕpaʁ}
\ipa{-r} \ipa{zdɔr} (\ipa{zdɯm}

\subsection{Nominalization}

\ipa{xcʰi} \ipa{ɣɟi}

 \section{Noun phrase}
\begin{table}[H]
\caption{Vowel fusion in Rtau nouns} \label{tab:alternation.noun} \centering
\begin{tabular}{c|cccccc}
\toprule
base form & meaning & ergative & genitive \\
\midrule
\ipa{kəta} & dog & \ipa{kətow} & \ipa{kətej} & \\
\ipa{vdzi} & man & \ipa{vdzu} & \ipa{vdzi} & \\
\ipa{xə} & hybrid of yak and cow & \ipa{xu} & \ipa{xi} & \\
\bottomrule
\end{tabular}
\end{table}

erg vs instr:


	ti ɮdow ŋa-ʁa tə-ɲcʰə (pas exprès)
	tu ɮda-kʰa ŋa-ʁa tə-ɲcʰə.

focus:
xə le kʰaχcə nə-ɟi-sə ŋə-rə.
045	rkəmə le təɲə nə-ŋɔ-sə ŋərə
\section{Relativization}

\section{Complementation}  

  ŋa ɲi-gi va-ɲə rɟev-ʁʚ ʁʚ
  I will help you take out the pigs


 ŋa ɕə-ba kʰã
 je dois y aller
\section{Classification} \label{sec:classification}
 
 
  \citet{lai14person}

\bibliographystyle{linquiry2}
\bibliography{bibliogj}
\end{document}
