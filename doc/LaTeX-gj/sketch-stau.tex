\documentclass[oneside,a4paper,11pt]{article} 
\usepackage{fontspec}
\usepackage{natbib}
\usepackage{booktabs}
\usepackage{xltxtra} 
\usepackage{lineno}
\usepackage{polyglossia}
%\usepackage[top=72pt,bottom=72pt,left=72pt,right=72pt]{geometry}
\usepackage[table]{xcolor}
\usepackage{gb4e} 
\usepackage{multicol,multirow}
\usepackage{graphicx}
\usepackage{float}
\usepackage{slashbox} 
\usepackage{rotating}
\usepackage{hyperref} 
\hypersetup{bookmarksnumbered,bookmarksopenlevel=5,bookmarksdepth=5,colorlinks=true,linkcolor=blue,citecolor=blue}
\usepackage[all]{hypcap}
\usepackage{memhfixc}
\usepackage{lscape}
\bibpunct[: ]{(}{)}{,}{a}{}{,}


\newfontfamily\phon[Mapping=tex-text,Ligatures=Common,Scale=MatchLowercase,FakeSlant=0.3]{Charis SIL} 
\newcommand{\ipa}[1]{{\phon #1}} %API tjs en italique
\newcommand{\ipapl}[1]{{\phon #1}} %API tjs en italique
 
\newcommand{\grise}[1]{\cellcolor{lightgray}\textbf{#1}}
\newfontfamily\cn[Mapping=tex-text,Scale=MatchUppercase]{MingLiU}%pour le chinois
\newcommand{\zh}[1]{{\cn #1}}

\newcommand{\jg}[1]{\ipa{#1}\index{Japhug #1}}
\newcommand{\wav}[1]{#1.wav}
\newcommand{\tgz}[1]{\mo{#1} \tg{#1}}

\XeTeXlinebreaklocale 'zh' %使用中文换行
\XeTeXlinebreakskip = 0pt plus 1pt %
 \newcommand{\ra}{$\Sigma_1$} 
\newcommand{\rc}{$\Sigma_3$} 
\newcommand{\ro}{$\Sigma$} 

\newcommand{\conv}{\textsc{conv}}
\newcommand{\dat}{\textsc{dat}}
\newcommand{\dem}{\textsc{dem}}
\newcommand{\evid}{\textsc{evd}}
\newcommand{\erg}{\textsc{erg}}
\newcommand{\inv}{\textsc{inv}}
\newcommand{\pfv}{\textsc{pfv}} 
\newcommand{\prf}{\textsc{pfv}}
\newcommand{\refl}{\textsc{refl}}
\newcommand{\sg}{\textsc{sg}} 
\newcommand{\testim}{\textsc{testim}}
 
\begin{document} 


\title{Stau}
\author{Guillaume Jacques, Lai Yunfan, Anton Antonov and Lobsang Nima}

\maketitle
%\linenumbers
 

\section{Introduction}
The cluster of languages variously referred to as Stau, Ergong or Horpa in the literature are spoken over a large area from Ndzamthang county (in Chinese Rangtang  \zh{壤塘县})  in Rngaba  prefecture (Aba \zh{阿坝州}) to Rtau county (Daofu \zh{道孚}) in Dkarmdzes prefecture (Ganzi \zh{甘孜州}), in sichuan province, China. At the moment of writing, it is still unclear how many unintelligible varieties belong to this group, but at least three must be distinguished: the language of Rtau county (referred as `Stau' in this paper), the Dgebshes language (Geshizha \zh{格什扎话})  spoken in Rongbrag county (Danba \zh{丹巴}), and the Stodsde language (Shangzhai \zh{上寨}) in Ndzamthang.  

Research on these languages is still limited. No dictionaries or text corpora have yet been published. Stodsde is only known by a few articles (\citealt{jackson00sidaba} and   \citealt{jackson07shangzhai}), Dgebshes by a grammatical sketch (\citealt{duoerji98geshizha}). The dialects spoken in Rtau county have been investigated by several teams of scholars (\citealt{huangbf91daofu}, \citealt{sun13gexi} and \citealt{jacques14rtau}), but no detailed description of this language is available.

There is no consensus as to how these languages should be named. The present paper favours the Tibetan names of these languages rather than the Chinese-based ones, since Chinese names are transcription of the Tibetan names. 

  The native speaker among the authors (Lobsang Nima) favours the name \ipa{rəsɲəske} for his language (the Khang.gsar \ipa{qʰərŋe} dialect in Rtau county, in Chinese \zh{孔色} Kongse), but this name is not used by all speakers and we prefer the geographically-based name  `Stau'. The spelling with \textit{St--} rather than \textit{Rt--}, aside from being more pronounceable for the average Western reader, better reflects the local pronunciation of the county name \ipa{stʚwu}. This spelling has already been used in English (see \citealt{wang70stau}).   Most of the data in these paper come from this dialect, except some Khroskyabs and Gyurong Horpa examples from Lai Yunfan's fieldwork.
  
From the group comprising all three languages (Stau, Dgebshes and Stodsde),  we can follow 
\citet{jackson00sidaba}'s term `Horpa' which has indeed been used for this area in the past, or the related `Tre-Hor'.  The term Ergong \zh{尔龚} used by scholars such as \citet{sun83liujiang}, on the other hand, appears to lack any basis in the local languages and should be avoided.
 
As shown in section \ref{sec:classification}, there is evidence from verbal morphology and lexicon that Horpa languages form a subgroup within Rgyalrongic with Khroskyabs (previously known by the incorrect term `Lavrung'), as these two branches present common innovations that are unlikely to be explainable as parallel developments.
 
\section{Phonology}
 
 
  \subsection{Onsets}
  Table \ref{tab:consonants} presents the consonantal inventory postulated for Stau. Unlike in Rgyalrong languages, there is no evidence for treating the prenasalized voiced stops as single phonemes in Stau.
  
 \begin{table}
 \caption{Consonantal phonemes in Stau} \label{tab:consonants}  \centering
 \resizebox{\columnwidth}{!}{
\begin{tabular}{llllllllll}
\toprule
& &Bilabial& Labiodental &Dental/Alveolar & Retroflex &Alveolo-palatal&Palatal & Velar&Uvular\\
\midrule
Stop&unvoiced & p & & t & & & c & k & q\\
&aspirated & pʰ  && tʰ & & & cʰ & kʰ & qʰ\\
&voiced & b & & d & & & ɟ & g &  \\
Affricate&unvoiced &  &  & ts & tʂ& tɕ &  &   &  \\
&aspirated &   && tsʰ & tʂʰ& tɕʰ &  &   &  \\
&voiced &   &  &dz & dʐ& dʑ &  &   &  \\
Nasal &&m& & n   &&&ɲ &ŋ \\
Fricative &unvoiced&  &f & s & ʂ& ɕ &  &x   &χ  \\
&voiced&&v & z & & ʑ &  &ɣ   &ʁ  \\
Approximant && w &&&  &&j &\\
Rhotic &&&&&r&&\\
Lateral &sonorant &&&l\\
&  fricative &&&ɬ\\
\bottomrule
\end{tabular}}
\end{table}

Some dialects of Stau have contrastive aspirated fricatives (see \citealt{jackson00puxi},  \citealt{jacques11lingua}). In the Khang.gsar dialect, the unvoiced fricative phonemes are realized as aspirated in syllable-initial position without a cluster, and as unaspirated when a cluster is present. Thus, \ipa{nə-sow} `I killed him' is realized as [nəsʰow], while no such as aspiration is observed in \ipa{nə-fse} `he killed him' due to the cluster.

Voiced stops are almost absent word-initially in the Khang.gsar dialect of Stau. In the case of verbs, the voiced stop of affricate resurfaces when a prefix is added, as in the examples in Table \ref{tab:voiced} (the bare stem appears in non-finite and in factual forms). In word-initial clusters whose second element is a voiced stop, voicing is preserved, as in \ipa{zgə} `cause to wear, put clothes on', a verb derived from \ipa{kə} `wear'. 

 \begin{table}[H]
 \caption{Neutralization of voiced stops in word-initial position} \label{tab:voiced} \centering 
\begin{tabular}{lllll}
\toprule
Bare stem & Meaning & Prefixed verb & Meaning \\
\midrule
\ipa{pərje} & burn (it) &\ipa{tə-bərje-sə} & it burnt \\
\ipa{te} & do &\ipa{nə-dej} & do it !\\
\ipa{tɕə} & meet &\ipa{kə-dʑõ} & I met (him)\\
\ipa{kə} &wear &\ipa{rə-gu} & I wore it \\
\bottomrule
\end{tabular}
\end{table}
Unvoiced stops and affricates initial consonants do not have any alternation, as for instance \ipa{tɕi} `wear (hat)' $\rightarrow$ \ipa{rə-tɕu} `I wore it'.

Nouns do not have any productive prefix, and thus the contrast has been lost, except for   \ipa{bəlʚ} `cheek', the only native word whose voicing is preserved word-initially.

 
The status of the unvoiced labiodental fricative \ipa{f} is unclear. This sound is never found as a simple onset in native words (only in borrowings from Chinese). However, some clusters such as  \ipa{ʂf} [\ipa{r̥ɸʷ}] (contrasting with \ipa{rv}), where \ipa{f} is the element closest to the vowel, are difficult to account for without postulating an unvoiced labial fricative phoneme.
 
 In clusters with a nasal as a first element, only stops and affricates are attested, except for the cluster \ipa{nɬ}, which is in free variation with \ipa{ntʰ}.
 
% \ipa{ɦwa} `hug'
  \subsection{Rhymes}
 
There are ten vowels in the Khang.gsar dialect of Stau, six plain vowels (--\ipa{i}, --\ipa{e}, --\ipa{a}, --\ipa{ə}, --\ipa{ʚ},  --\ipa{u}), two velarized vowels (--\ipa{oˠ}, --\ipa{aˠ}) and two nasal vowels (--\ipa{õ}   --\ipa{ã}).

The velarized vowels are almost exclusively attested in Tibetan loanword (--\ipa{oˠ} corresponds to Tibetan \ipa{--og} and --\ipa{aˠ} to \ipa{--ag} and \ipa{--eg}), but there are a few native words with velarized vowels too, such as \ipa{mbjoˠ} `fast' (cognate to Japhug \ipa{mbjom} `fast').

Only three codas are possible in Stau: \ipa{--v},  \ipa{--r} and  \ipa{--m}, the latter only attested in Tibetan loanwords. In native words, the coda \ipa{--r} sometimes appear to correspond  to \ipa{--r} in other Rgyalrongic languages (as in \ipa{χtɕʰər} `sour', Japhug \ipa{tɕur}), but in other cases, such as \ipa{spar} `be thirsty' (Japhug \ipa{ɕpaʁ}) or \ipa{zdʚr}  cloud, be cloud' (Japhug \ipa{zdɯm} `cloud'), there is no \ipa{--r} coda anywhere else in the family. A possible explanation is that the testimonial suffix \ipa{--rə}, whose vowel tend to be elided, has been reanalyzed as part of the stem in third person forms.

 \subsection{Vowel fusion}
 In order to account for vowel alternations observed in the verbal and nominal systems, \citet{jacques14rtau} postulate a series of vowel fusion rules, summarized in \ref{tab:alternation} (the symbol C stands for either final \ipa{--r} or final \ipa{--v}).
\begin{table}[H]
\caption{Vowel fusion in Stau} \label{tab:alternation} \centering
\begin{tabular}{llllllll}
\toprule

 \backslashbox{Stem}{Suffix} &  	--\ipa{w} &  --\ipa{ã} &  --\ipa{j} \\
\hline
\ipa{i}&\ipa{u}&\ipa{ã}&\ipa{i}\\
\ipa{e}&\ipa{ow}&\ipa{ã}&\ipa{ej}\\
\ipa{a}&\ipa{ow}&\ipa{ã}&\ipa{ej}\\
\ipa{ə}&\ipa{u}&\ipa{õ}&\ipa{i}\\
\ipa{ʚ}&\ipa{ow}&\ipa{õ}&\ipa{ej}\\
\midrule
\ipa{əv}&\ipa{u}&\ipa{õ}&\ipa{iv}\\
\ipa{ər}&\ipa{u}&\ipa{õ}&\ipa{i}\\
\ipa{a}C&\ipa{ow}&\ipa{ã}&\ipa{ej}\\
\ipa{e}C&\ipa{ow}&\ipa{ã}&\ipa{ej}\\
\ipa{ʚ}C &\ipa{ow}&\ipa{õ}&\ipa{ej}\\
\bottomrule
\end{tabular}
\end{table}


Vowel fusion cannot occur with the vowels -\ipa{u}, --\ipa{oˠ}, --\ipa{aˠ}, --\ipa{õ} and --\ipa{ã} and the rhymes ending in \ipa{--m}.
 
\section{Verbal morphology}


\subsection{Intransitive conjugation}
In the intransitive conjugation, the third and second persons singular forms are in the bare stem, while first person (singular and plural) forms have a suffix \ipa{--ã}.

Following the rules of  vowel fusion in Table \ref{tab:alternation}, vowel fusion between the first person suffix \ipa{--ã} and the verb stem has differ results depending on the vowel (and coda) of the verb stem. 
 
Six classes of alternations are found in verbs with open syllables; class 6 includes verbs without alternation, whose rhyme can be any of --\ipa{u}, --\ipa{oˠ}, --\ipa{aˠ}, --\ipa{õ} and --\ipa{ã}:
\begin{table}[H]
\caption{Vowel alternations in open-syllable intransitive verbs in Rtau} \label{tab:open.intr} \centering
\begin{tabular}{llll|ll|l}
\toprule
%Verb class&1&2&3&4&5&6 \\
Meaning &	look at   &  	move   &  	like&  	be full     &  	 	be ill      &  	be hot       \\  
\midrule
1&	\ipa{scəqã} & 	\ipa{mbəɕã} & \ipa{rgã} &	\ipa{fkõ} & 	  	\ipa{ŋõ} & 	   	\ipa{cʰu}   \\ 
1 (underlying)&	\ipa{scəqi-ã} & 	\ipa{mbəɕe-ã} & \ipa{rga-ã} &	\ipa{fkə-ã} & 	  	\ipa{ŋɞ-ã} & 	   	\ipa{cʰu-ã}   \\ 
2/3&	\ipa{scəqi} & 	\ipa{mbəɕe} & \ipa{rga} & 	\ipa{fkə} & 	  	\ipa{ŋɞ} & 	 	\ipa{cʰu}  \\ 
\bottomrule
\end{tabular}
\end{table}


There are two irregular verbs with intransitive morphology  which have \ipa{--ã} in the first person instead of expected \ipa{--õ} (cf table \ref{tab:irr.intr}). 

\begin{table}[H]
\caption{Irregular intransitive verbs in Stau} \label{tab:irr.intr} \centering
\begin{tabular}{llllll}
\toprule
meaning &	go     & say &\\  
\midrule
1&	\ipa{ɕã}  	 &\ipa{jã} &\\ 
2/3&	\ipa{ɕə} & 	\ipa{jə} &\\ 
\bottomrule
\end{tabular}
\end{table}

 \subsection{Transitive conjugation}
The basic structure of transitive conjugations  in Khang.gsar Stau can be illustrated by Table \ref{tab:kill} (the columns represent the P and the lines the A). In addition to the first person suffix \ipa{--ã} , we find two additional suffix (\textsc{1sg}$\rightarrow$3  \ipa{--w} and 2$\rightarrow$3 \ipa{--j}) and the inverse prefix \ipa{f--/v--}. The only unmarked form in the paradigm is the 1$\rightarrow$2 slot, a curious fact which however can be accounted for historically (see \citealt{jacques14rtau}).



\begin{table}[h]
\caption{The transitive conjugation in Stau}
\centering \label{tab:kill}
\begin{tabular}{|c|cc|c|c|}  
 \cline{1-5}
\backslashbox{A}{P} &\textsc{1sg}  &  \textsc{1pl}  &  2  &  	3  \\  
\cline{1-5}
 \textsc{1sg}  &  	 \multicolumn{2}{c}{\cellcolor{lightgray}}   \vline    &  	\multirow{2}{*}{\ro{}}  &  	\ro{}\ipa{-w}  \\  
\cline{5-5}\textsc{1pl}  &  \multicolumn{2}{c}{\cellcolor{lightgray}} 	 \vline   &   &  	\ipa{\ro{}-ã}  \\  
\cline{5-5}2 &    \multicolumn{2}{c}{\multirow{2}{*}{\ipa{v-\ro{}-ã}}}    \vline  &   \grise{ }	  &  	\ipa{\ro{}-j}  \\  
\cline{5-5}3 &  \multicolumn{2}{c}{ } \vline &  	\multicolumn{2}{c}{ \ipa{v-\ro{}}}   	 \vline  \\  
\cline{1-5}
\end{tabular}
\end{table}

The vowel fusion rules in Table \ref{tab:alternation} apply to all suffixed forms, as in the intransitive paradigms. No example of irregular vowel fusion has yet been discovered with transitive verbs.

The \ipa{f}-- / \ipa{v}-- prefix appears in 2/3$\rightarrow$1, 3$\rightarrow$2 and 3$\rightarrow$3 forms. Its presence  in 2$\rightarrow$1 precludes an analysis as a third person agent marker, and it is best to treat it as an inverse marker, although there is no obviation contrast in  3$\rightarrow$3 forms (see section \ref{sec:classification}). The inverse prefix does not occur with verb stems containing an initial cluster, or with monosyllabic verbs with the coda \ipa{--v} such as \ipa{kʰev} `scoop (water)' or \ipa{ɕev} `take out (of a pile)'. 

 \subsection{Other dialects}
Some dialects of Stau have person marking systems that are markedly different from the one described above. In the Gyurong (\ipa{ʁjɯrõ}) variety, for instance, we find the paradigms indicated in Table \ref{tab:gyurong}.
 
\begin{table}[h]
\caption{Person indexation in the Gyurong dialect}
\centering \label{tab:gyurong}
\begin{tabular}{|c|c|c|c|c|}  
 \cline{1-5}
\backslashbox{A}{P} &\textsc{1sg}  &  \textsc{1pl}  &  2  &  	3  \\  
\cline{1-5}
 \textsc{1sg}  &  	 \multicolumn{2}{c}{\cellcolor{lightgray}}   \vline    &  	\multirow{2}{*}{\ro{}\ipa{-n}}  &  	\ro{}\ipa{-ŋ}  \\  
\cline{5-5}\textsc{1pl}  &  \multicolumn{2}{c}{\cellcolor{lightgray}} 	 \vline   &   &  	\ipa{\ro{}-j}  \\  
\cline{5-5}2 &    	\multirow{2}{*}{\ipa{v-\ro{}-ŋ}}   	&	\multirow{2}{*}{\ipa{v-\ro{}-j}}   	 &   \grise{ }	  &  	\ipa{\ro{}-n}  \\  
\cline{5-5}3 &     	&    	 &  	  \ipa{v-\ro{}-n}   	& \ipa{v-\ro{}}   \\  
\cline{1-5}
\textsc{intr} & \ro{}\ipa{-ŋ} & \ro{}\ipa{-j} &\ro{}\ipa{-n}  &\ro{} \\
\cline{1-5}
\end{tabular}
\end{table}
In comparison with Khanggsar, the Gyurong dialect presents both innovative and conservative  features. Khanggsar is more conservative in having a special marker \ipa{-w} for first singular direct, distinct from the first singular inverse, whereas Gyurong, like Khroskyabs and Rgyalrong languages, has the suffix in \textsc{1sg} intransitive, \textsc{1sg}$\rightarrow$3 and 2/3$\rightarrow$\textsc{1sg} forms. Gyurong on the other hand preserves a distinction between first singular \ipa{--ŋ} and plural \ipa{--j} in all forms, and has the second person suffix \ipa{--n} even in the intransitive second person, the 3$\rightarrow$2 and the 1$\rightarrow$2 forms.

\subsection{Directional and negative prefixes}
In Stau, directional prefixes come in two series (Table \ref{tab:dir.pref}),  the  \ipa{-ə} series for imperative and perfective/evidential, and the \ipa{--í--} series for interrogative and irrealis forms. A few verbs few verbs such as  \ipa{vdʚ} `see' \ipa{ste} `finish' or \ipa{si} `know' never appear with any directional prefix even in the perfective.

\begin{table}[h]
\caption{Directional prefixes in Stau} 
\label{tab:dir.pref} \centering
\begin{tabular}{llllllll}
\toprule
Direction & Perfective / Imperative & Interrogative / irrealis \\
\midrule
 Up & \ipa{rə--} & \ipa{rí--} \\
Down & \ipa{nə--} & \ipa{ní--} \\
North  & \ipa{kə--} & \ipa{kí--} \\
South & \ipa{ɣə--} & \ipa{ɣí--} \\
No direction & \ipa{tə--} & \ipa{tí--} \\
\bottomrule
\end{tabular}
\end{table}

There are three negative prefixes in Stau: the past negative \ipa{ma--}, non-past \ipa{mí--}, and the prohibitive \ipa{di--}. The prohibitive must be used with a directional prefix, as in   \ipa{nə-di-f-se-ã}  (\ipa{nədifsã}) \textsc{pfv-prohib-inv}-kill-1 `don't kill me'. The past negative can be used with or without the directional (see \ref{ex:nEmavu}). In both cases, the negative appears after the direction prefix, the reversed other of Rgyalrong languages.


\begin{exe}
\ex \label{ex:nEmavu} 
\gll
\ipa{e-cʰe} 	\ipa{nə-ma-və-w}  \\
one-\textsc{cl} \textsc{pfv-neg}-do-\textsc{1sg}$\rightarrow$3 \\
\glt I did not do anything.
\end{exe}


 \subsection{Derivational morphology}
 
The only denominal prefix in Stau is \ipa{s--/z--}, cognate of Japhug \ipa{sɯ--/sɤ--/ɕɯ--} (on which see \citealt[14-17]{jacques14antipassive}). It derives transitive verbs, illustrated by the examples in Table \ref{tab:denominal}. While the original semantics of this prefix was likely `use X' and `cause sb. to have X' as in Japhug, in Stau the semantics of the denominal verbs are largely unpredictable, and result from semantic shifts (`use a staff' $\rightarrow$ `hit with a staff' $\rightarrow$  `hit'). The only denominal verb shared by Stau, Khroskyabs and Japhug is \ipa{smi} `give a name' (in Japhug \ipa{sɤrmi}).
 
 \begin{table}[H]
 \caption{Denominal verbs in Stau} \label{tab:denominal} \centering 
\begin{tabular}{lllll}
\toprule
Base noun & Meaning & Denominal verb & Meaning \\
\midrule
\ipa{pəcʰa} & staff&\ipa{zbəcʰa} & hit \\
\ipa{rmi} &name &\ipa{smi} &give a name \\
\ipa{ɣme} &wound &\ipa{smi} & hurt \\
\bottomrule
\end{tabular}
\end{table}
 
 
 Stau has a few example of anticausative verbs with voiced onset (Table \ref{tab:anticausative}). The anticausative verbs derive from the transitive ones (not the opposite direction of derivation, see \citealt{jacques12demotion}). There is an irregularity in the pair \ipa{ftɕə} vs \ipa{dʑə} `melt tr/it', as the \ipa{f--} element of the transitive verb has no equivalent in the anticausative one. Interestingly, the same irregularity is found in the cognate pair in Japhug (\ipa{ftʂi} vs \ipa{ndʐi}).
 
Unlike Rgyalrong languages, anticausative derivation in Stau is attested in verb with fricative initial consonants (as in \ipa{zəla}  `fall').
 
  \begin{table}[H]
 \caption{Anticausative verbs in Stau} \label{tab:anticausative} \centering 
\begin{tabular}{lllll}
\toprule
Base verb & Meaning & Anticausative verb & Meaning \\
\midrule
 \ipa{səla} &cause to fall & \ipa{zəla}  & fall \\
\ipa{pʰre}   &break (tr) &\ipa{bre}   & break (it) \\
  \ipa{fkʰe} & cut down & \ipa{vge} & break away, off \\
\ipa{ftʂə}& wake (tr) & \ipa{brə}& wake up\\
\ipa{ftɕə} &melt (tr) &\ipa{dʑə} &melt (it)\\
 \bottomrule
\end{tabular}
\end{table}
 
There is a causative \ipa{s--/z--} prefix in Stau attested in a few verbs, but unlike in Rgyalrong languages, it is not productive. Table \ref{tab:causative} provides a representative list of verb pairs. The phonological alternations attested with the \ipa{s--/z--} prefix are much less complex than those attested in Stodsde (\citealt{jackson07shangzhai}) or in Khroskyabs (\citealt{lai14caus}). However, the causative forms are not always predictable from the base form: for instance, the causative form of voiced initial verbs can be either voiced (\ipa{zgə} `put clothes on') or unvoiced (\ipa{spərje} `burn (tr)').
 
 
   \begin{table}[H]
 \caption{Causative verbs in Stau} \label{tab:causative} \centering 
\begin{tabular}{lllll}
\toprule
Base verb & Meaning & Causative verb & Meaning \\
\midrule
 \ipa{lə}  &boil& \ipa{zɮdə}& boil (tr)\\
  \ipa{cʰu}  & hot &  \ipa{scʰu}& cook \\
    \ipa{rŋi}  & borrow &  \ipa{sŋi}& lend \\
    \ipa{qə}  & go out (fire) &  \ipa{sqʰə}& extinguish\\
    \ipa{tʰi}  & drink &  \ipa{stʰi} & give to drink\\
    \ipa{kə} / \ipa{-gə} & wear &  \ipa{zgə} & put clothes on \\
     \ipa{nə}  & burn (it) &  \ipa{snə}& burn (tr)\\
     \ipa{pərje} / \ipa{-bərje} & burn &  \ipa{spərje} & burn (tr)\\
 \bottomrule
\end{tabular}
\end{table}

Some Tibetan  loanwords, borrowed in pairs, should be distinguished from the native causative pairs (\ipa{mbjer} `be pasted on' vs \ipa{zɟwer}  `paste' from Tibetan \ipa{ɴbʲar} and \ipa{sbʲar}).
 
 
 In addition to the prefix \ipa{s-/z-}, there is one example of a causative \ipa{ɣ} prefix in the pair   \ipa{ndʑi} `learn'    $\rightarrow$  \ipa{ɣʑi} `teach'. This may be a fossilized allomorph of the causative prefix (in Khroskyabs the corresponding pair is \ipa{ndzé} `learn', \ipa{ldzê}  teach' with \ipa{l--} allomorph of the causative prefix).
     
The \ipa{s--/z--} has semantic effects that are sometimes better described with terms other than `causative'. There is one example of applicative \ipa{s--} prefix: \ipa{qʰe}  `laugh' $\rightarrow$ \ipa{sqʰe} `laugh at. In addition, we find \ipa{nənʚ}  `smell (it)' $\rightarrow$ \ipa{snəsnʚ} `smell (tr)' (not `cause to have a smell'), whose semantics can be described as `tropative' (see \citealt{jacques13tropative}).
% zja?
 
 
 There is little evidence of incorporation in Stau, unlike Japhug (\citealt{jacques12incorp}) or Khroskyabs (\citealt{lai13affixale}, \citealt{lai14person}). Only two examples are found: \ipa{rvatɕa} `carry on shoulders' (which incorporates the noun \ipa{rva} `shoulders and upper back') and \ipa{mbarji} `stride over' (with \ipa{pa}  `a step').

 

 \section{Noun phrase}
 
There are five postpositions in Stau: the ergative \ipa{--w}, the genitive \ipa{--j}, the instrumental \ipa{--kʰa}, the datif \ipa{--gi} and two locatives \ipa{--tɕʰa} and \ipa{--ʁa}.


The ergative and the genitive merge with the last word of the preceding noun phrase, and the regular vowel fusion rules seen in table \ref{tab:alternation} apply. Table \ref{tab:alternation.noun} presents example of ergative and genitive forms of some common nouns.

\begin{table}[H]
\caption{Vowel fusion in Rtau nouns} \label{tab:alternation.noun} \centering
\begin{tabular}{l|lllllll}
\toprule
base form & meaning & ergative & genitive \\
\midrule
\ipa{kəta} & dog & \ipa{kəta-w} $\rightarrow$ \ipa{kətow} & \ipa{kəta-j} $\rightarrow$ \ipa{kətej} & \\
\ipa{vdzi} & man & \ipa{vdzi-w} $\rightarrow$  \ipa{vdzu} & \ipa{vdzi-j} $\rightarrow$ \ipa{vdzi} & \\
\ipa{xə} & hybrid of yak and cow &\ipa{xə-w}  $\rightarrow$\ipa{xu} & \ipa{xə-j} $\rightarrow$ \ipa{xi} & \\
\bottomrule
\end{tabular}
\end{table}

The first and second person pronouns (\ipa{ŋa} \textsc{1sg}, \ipa{ɲə} \textsc{2sg}) do not have ergative forms (except in hybrid indirect speech, see below), as illustrated by example \ref{ex:qow}, where the noun \ipa{waqi-w}  `the rabbit' takes the ergative while the pronoun \ipa{ŋa} remains invariable in exactly the same context.
 \begin{exe}
\ex \label{ex:qow}
\gll
\ipa{ŋa} 	\ipa{zəŋʚ} 	\ipa{qe-w} \ipa{tə-jə-sə} 	\ipa{ŋə-rə.} \\
\textsc{1sg} first shoot-\textsc{1sg$\rightarrow$3} \textsc{pfv}-say-\textsc{evd} be-\textsc{testim}\\
\gll \ipa{tɕʰəge,} 	\ipa{waqi-w} 	\ipa{zəŋʚ} 	\ipa{tə-f-qe-sə} 	\ipa{ŋə-rə.} 	\\
then rabbit-\textsc{erg} first \textsc{pfv-inv}-shoot-\textsc{evd} be-\textsc{testim}\\
\glt He said `I will shoot first'. Then, the rabbit shoot first. (the rabbit and the tiger, 21-22)
\end{exe}

Some morphologically intransitive verbs, such as \ipa{rga} `like' can take an argument with the ergative, as in \ref{ex:rga} (the experiencer is in the ergative, and the stimulus in the dative). Morphological and syntactic transitvity in Stau are thus distinct and do not necessarily match for all verbs.

 \begin{exe}
\ex \label{ex:rga}
\gll \ipa{tə-w}  	\ipa{ŋa-ɡi}  	\ipa{rɡa-rə}  
 \\
he-\textsc{erg} I-\textsc{dat} like-\textsc{testim} \\
\glt `(S)he likes me.'
\end{exe}

All postpositions above can occur in the argumental structure of some verbs, including the instrumental \ipa{--kʰa}, which is selected by the verb \ipa{mkʰʚ} `need, want', as in \ref{ex:mimkhoN}.

 \begin{exe}
\ex \label{ex:mimkhoN}
\gll
\ipa{ŋa} 	\ipa{rji-kʰa} 	\ipa{mí-mkhʚ-ã} \\
\textsc{1sg} horse-\textsc{instr} \textsc{neg:n.pst}-want-1 \\
\glt I don't want a horse.
\end{exe}

%	ti ɮdow ŋa-ʁa tə-ɲcʰə (pas exprès)
%	tu ɮda-kʰa ŋa-ʁa tə-ɲcʰə.
%	

%	ŋa rji tɕəma : je n'ai pas besoin d'un cheval

 


%xə le kʰaχcə nə-ɟi-sə ŋə-rə.
%045	rkəmə le təɲə nə-ŋʚ-sə ŋərə
%
%		tə ge vdzi ke ʁje gə ŋərə (ge: positif)
%		tə le vdzi ke mtsʰer gə ŋərə (le: négatif)
%
%
%contrastif
%ŋa sə rdʑi-ʁa tɕʰu χa mi-gõ-rə 
%
%vʚ sə ɮdə ja, ɣrə de ə́-ɮdə mí-ɮdə χa mi-gõ
%
%
%ŋa zəŋʚ qe-ŋkʰə ŋõ ŋõ va ŋõ
%C'est moi qui tire en premier

%zəŋʚ fqe-ŋkʰə staˠ ŋə-rə
%staˠ zəŋʚ qe-ŋkʰə ŋʚŋʚ va ŋə-rə
%
%akəstʚmbow ləla fcʰə fcʰə va ŋə-rə


\section{Nominalization} \label{sec:nmlz}
All productive nominalization markers in Stau are suffixes. They are commonly used to build relative and complement clauses.

We find the agentive nominalizer \ipa{--ŋkʰə}, that can be used with intransitive (\ipa{ɕə}  `go' $\rightarrow$ \ipa{ɕə-ŋkʰə} `the one who goes') as well as transitive verbs (\ipa{rə} `buy' $\rightarrow$ \ipa{rə-ŋkʰə} `buyer'). The verb loses all person morphology (including the inverse prefix -- note that the third person form of \ipa{rə} `buy' is \ipa{vrə} `he buys') and only the negative prefixes can be added.
 
The nominalizer \ipa{--lə} can be used with transitive verbs to designate the patient of the action, as in \ipa{fɕe} `tell' $\rightarrow$ \ipa{fɕe-lə} `things that have been told' or \ipa{ŋgə} `eat' $\rightarrow$ \ipa{ŋgə-lə} `food'. Alternatively, it can build action nominals with either transitive or intransitive verbs, as  in example \ref{ex:tEmphjEsE}.

\begin{exe}
\ex \label{ex:tEmphjEsE} 
\gll
\ipa{ɲə-ɲə} 	\ipa{nə-vi-lə} 	\ipa{de} 	\ipa{tə-mpʰjə-sə} 	\ipa{ŋə-rə} \\
2-\textsc{pl} \textsc{down}-go-\textsc{nmlz:action} \textsc{dem} \textsc{pfv}-be.late-\textsc{evd} be-\textsc{testim}\\
\glt You arrived there too late (=your arrival there was late; Akhu stonba and the Walnut tree, 34).
\end{exe}

The suffix \ipa{--re} is used for locative nominalization, as in \ipa{rkʰə} `put in' $\rightarrow$ \ipa{pʰjodzə} \ipa{rkʰə-re} (money put.in-\textsc{nmlz:locative}) `wallet, place one puts money in'. It also appears with stative existential verbs such as \ipa{tə} `be there, exist' $\rightarrow$ \ipa{tə-re} (be.there-\textsc{nmlz:locative}) `the place where (it) is'. 

Finally, \ipa{--sce} is the marker of instrumental nominalization, as in \ipa{ra} `write' $\rightarrow$\ipa{ra-sce} `pen, the thing one uses to write'.

Relatives in Stau are always built using one of these four suffixes. The S/A  argument is relativized with \ipa{-ŋkʰə}, the P argument with \ipa{-lə}, and oblique arguments or adjuncts with the other two suffixes. 
%tə lɔrɟə de sɔŋɔ qʰana ɲə gi rgə-tɕʰa fɕe-lə ŋə-rə

In relative clauses with relativized P, the agent of the relative can be marked either with the ergative or the genitive. In example \ref{ex:bden}, it is possible to say  \ipa{akəstʚmba-w} with the ergative \ipa{-w} instead.

\begin{exe}
\ex \label{ex:bden}
\gll 
 [\ipa{akəstʚmba-j} 	\ipa{fɕe-lə}] 	\ipa{de} 	\ipa{vdẽ-ndʐə?} \\
 Akhustonba-\textsc{gen} tell-\textsc{nmlz:P} \textsc{dem} be.true-\textsc{testim} \\
\glt Does Akhustonba tell the truth?  (`It is true what Akhustonba says', Akhu stonba and the Walnut tree, 25)
\end{exe}

 
 

\section{Complementation}  
Most complement clauses have a non-finite verb, suffixed with the \ipa{-lə} or \ipa{--re} nominalizers.

The modal verb \ipa{rə} `be able' is an example of the first type, as shown by example \ref{ex:marusE}, where the verb of the complement clause appears without person marking and suffixed with \ipa{--lə}.

\begin{exe}
\ex\label{ex:marusE}
\gll
\ipa{tɕʰəge} 	\ipa{ŋa} 	\ipa{le} 	\ipa{rõsa} 	\ipa{scəqi-lə} 	\ipa{ma-rə-w-sə} \\
then \textsc{1sg} \textsc{top} immediately look.at-\textsc{nmlz:P/action} \textsc{neg:pfv}-be.able-\textsc{1sg}$\rightarrow$3-\textsc{evd} \\
\glt I was not able to notice it immediately. (The hybrid yak, 52)
\end{exe}

This type of complement is also found with the phasal complex predicate \ipa{ŋgʚ} \ipa{ftsu} `start', borrowed from Tibetan Tibetan \ipa{ɴgo btsugs} `start' (see example \ref{ex:btsugs}).

\begin{exe}
\ex \label{ex:btsugs}
\gll 
\ipa{ŋa} 	\ipa{zɮdə-lə} 	\ipa{ŋgʚ} 	\ipa{kə-ftsu-w} \\
\textsc{1sg} boil-\textsc{nmlz:P/action} start \textsc{pfv}-start \\
\glt I started boiling it.
\end{exe}

The locative nominalizer \ipa{--re} appears for instance in combination with \ipa{ɮdi} `come' to express the meaning `be ready to, have almost ...', as in \ref{ex:stere}.

\begin{exe}
\ex \label{ex:stere}
\gll 
 \ipa{ste-re} 	\ipa{ɮdi-ã} \\
finish-\textsc{nmlz:locative} come-1 \\
\glt I have almost finished.
\end{exe}

% 		ɮde-re ɮdi-rə : on était presque arrivé 
%		ste-re ɮdã: j'ai presque fini
%		ŋgə-re ɮdã: je suis prêt pour manger


Other types of non-finite complement clauses are also attested. The verb \ipa{ʁʚ}  `help' takes a complement clause  whose main verb is marked with the suffix \ipa{--ʁʚ} (homophonous with the bare stem of the verb  help'), as in \ref{ex:kERoN}. No other verb can occur with this type of complement clauses.

\begin{exe}
\ex \label{ex:kERoN}
\gll 
\ipa{tə-w} 	\ipa{ŋa-gi} 	\ipa{tɕədə} 	\ipa{de} 	\ipa{zɮda-ʁʚ} 	\ipa{kə--ʁʚ-ã} \\
\textsc{3-erg} \textsc{1sg-dat} book \textsc{dem} read-\textsc{conv} \textsc{pfv-inv:}help-1 \\
\glt He helped me reading this book.
\end{exe}


In the synthetic causative construction, the verb \ipa{xtʂʰə} `let' take a complement verb in bare stem form, without any person or TAM marker, and no nominalization or converbial suffix. Example \ref{ex:tExtsxhoN} illustrates this construction. If the verb in the complement clause is intransitive, its S is coreferent with the P of \ipa{xtʂʰə} `let'.

\begin{exe}
\ex \label{ex:tExtsxhoN}
\gll 
[\ipa{e-ʑe} 	\ipa{nə}] 	\ipa{tə-xtʂʰə-ã} \\
one-moment rest \textsc{imp-inv:caus}-1 \\
\glt Let me rest a moment. (The hybrid yak, 48)
\end{exe}


%
%tɕədə de χte-ba tə-kʰã:
% ŋa ɕə-ba kʰã
% je dois y aller
% 
% 		ɮde-re ɮdi-rə : on était presque arrivé 
%		ste-re ɮdã: j'ai presque fini
%		ŋgə-re ɮdã: je suis prêt pour manger
%		
%
%
%		sə ma-ɮdi ga tə-ntsʰu: je pense à ceux qui étaient venus.
%		sə ɮdi-ga ŋa mi-dʐẽ-rə: Je ne me souviens plus de qui était venu.
%		
%		tɕʰəge ŋa le rõsa scəqi-lə ma-ru-sə 
%		
%		ŋa zɮdə-lə ŋgʚ kə-ftsu

Finite complement clauses are  found with both phasal auxiliary verbs such as \ipa{ste}  `finish' (as in \ref{ex:xtse}), and in reported speech.
\begin{exe}
\ex \label{ex:xtse}
\gll
\ipa{ɲəqʰej} 	\ipa{xtse} 	\ipa{ŋa} 	\ipa{tə-tʰi-w}]	\ipa{ste-sə} \\
\textsc{2sg:gen} soup \textsc{1sg} \textsc{pfv}-drink-\textsc{1sg$\rightarrow$3} finish-\textsc{evd} \\
\glt I drank up your soup. (The louse and the flea, 9)
\end{exe}

%055	rkə tə-və-si skarva (fɕaχpa) de ti ɣɔrɔ kʰɔ-ba tə-kʰi-sə ŋərə.
%\b	Il dut lui donner cela en dédommagement de son vol. (ti = rkəmi, təgi =rõba-gi)

Stau presents an extreme case of hybrid indirect(or semi-direct)  speech  (\citealt{aikhenvald08semidirect}, \citealt{tournadre08conjunct}). 

In case when the original speaker is different from the current speaker, the reported utterance will only be identical to the original utterance if the current and original speaker and addressee are not referred to in it. Thus, in example \ref{ex:1b}, the reported utterance is identical to the original one (example \ref{ex:33}).


\begin{exe}
\ex \label{ex:1b}
\gll
	\ipa{tʂaɕi-w}$_i$ \ipa{ŋa-gi}	\ipa{jə-rə-ge} [\ipapl{tə-w}$_j$	\ipapl{dʐʚma} 	\ipapl{de} \ipapl{nə-f-se-sə}] \ipa{jə-rə}  \\
	{Bkrashis-\erg} {1\sg-\dat} {say-\testim-\conv} {3\sg-\erg} Sgrolma {\dem} {\prf-\inv-kill-\evid}  say-\testim\\ 
	\glt Bkrashis$_i$ told me (that) he$_j$ ($\ne$ Bkrashis) had killed Sgrolma.
\end{exe}


\begin{exe}
\ex \label{ex:33}
\gll
	\ipa{tə-w} \ipapl{dʐʚma} \ipapl{de} \ipapl{nə-f-se(-sə)} \\
	{3\sg-\erg} Sgrolma {\dem} {\prf-\inv-kill(-\evid)}\\ 
	\glt He killed Sgrolma.
\end{exe}

A different situation is observed when the original speakers refers to himself (in the first person) in the original utterance, as illustrated by \ref{ex:1c} and \ref{ex:EdEgi}.  The reported utterance keeps the verb form of the original utterance (example \ref{ex:13}), but the first person pronoun is replaced by the logophoric pronoun \ipa{ədə}. Like the first person pronoun \ipa{ŋa}, the logophoric \ipa{ədə} does not take the ergative marker.

\begin{exe}
\ex \label{ex:1c}
\gll
	\ipa{tʂaɕi-w}$_i$ \ipa{ŋa-gi}	\ipa{jə-rə-ge} [\ipapl{ədə}$_i$	\ipapl{dʐʚma} 	\ipapl{de} 	\ipapl{nə-se-w}] \ipa{jə-rə}  \\
	{Bkrashis-\erg} {1\sg-\dat} {say-\testim-\conv} \textsc{logophoric} Sgrolma {\dem} {\prf-kill-1\sg$\rightarrow$3} say-\testim\\ 
	\glt Bkrashis$_i$ told me that he$_i$ had killed Sgrolma.
\end{exe}


\begin{exe}
\ex \label{ex:13}
\gll
	\ipa{ŋa}	\ipapl{dʐʚma} 	\ipapl{de} 	\ipapl{nə-se-w} \\
	{1\sg} Sgrolma {\dem} {\prf-kill-1\sg$\rightarrow$3}\\ 
	\glt I killed Sgrolma.
\end{exe}


\begin{exe}
\ex \label{ex:EdEgi}
\gll
		[\ipa{ədə-gi} 	\ipa{mbjoˠmbjoˠ} 	\ipa{tɕa} 	\ipa{gə} 	\ipa{tə-f-kʰʚ-ã,} 	\ipa{tɕa} 	\ipa{gə} 	\ipa{tə-f-kʰʚ-ã,}] 	\ipa{jə-rə} 	\\
\textsc{logophoric-dat} quickly tea \textsc{indef} \textsc{imp-inv}-give-1 tea \textsc{indef} \textsc{imp-inv}-give-1 say-\textsc{testim} \\
\glt He said `Give me some tea, give me some tea.' (The hybrid yak, 39)
\end{exe}


When the current speaker was referred as a third person in the original utterance, a different type of mismatch occurs, illustrated by example \ref{ex:1d}. The verb  preserves the form of the original utterance (example \ref{ex:33}), but the third person pronoun is replaced by the first person \ipa{ŋa}. In addition in this case, the pronoun takes the ergative flagging \ipa{--w} of the original utterance, although SAP pronouns normally do not take ergative suffixes.


\begin{exe}
\ex \label{ex:1d}
\gll
	\ipa{tʂaɕi-w}	\ipa{jə-rə-ge} [\ipa{ŋa-w} \ipa{dʐʚma} \ipa{de} \ipa{nə-f-se-sə}] \ipa{jə-rə}  \\
	{Bkrashis-\textsc{erg}} {say-\textsc{testim-conv}} {\textsc{1sg}-\textsc{erg}} Sgrolma {\textsc{dem}} {\textsc{pfv-inv}-kill-\textsc{evd}} say-\textsc{testim} \\
	\glt Bkrashis said that I had killed Sgrolma.
\end{exe}

We see that in Stau hybrid indirect speech, verb forms represent the point of view of the original speaker, but everything else, including pronouns and their case marking, represent that of the current speaker.


%	>	ɲəqʰʚ rə-ɮde ŋge, χa kʚ ŋərə
%	>	viens par toi même et tu verras.
	

	
%ɲi ŋa-gi kə-ma-ʁʚ ve, ŋa liska kʰərma gə və mí-vzu-rə
%Si tu ne m'aides pas, je ne pourrai pas faire ce travail tout seul
%
%ɲi ŋa-gi kə-ʁʚ demdi (sə), ŋa liska de və ma-vzu.



%	ɲi ŋa ʁa nə-ŋʚ ve, sʚ cʰe-rə
%	ɲi ke cʰe-me ŋʚ-rə
%	You are  bigger 
%	
%	liska ɕə-ʁa nəŋʚve, ɲchara təɕã ve sʚ scə
%	Ce serait plus sympa que j'aille me promener plutôt que d'aller travailler
%	
%jitʰa ndzʚ-ʁa nə-ŋʚ-ve, smẽba qʰi tə-ɕə ve, sʚ xtɕʰi ŋə-rə	
%	
%	ŋa ɲi nəŋʚve, liska ɕã
%SI j'étais toi, j'irai travailler	
\section{Evidentiality}

Like most languages of the Tibetan cultural area, the Stau verbal system has evidential markers, in particular the testimonial \ipa{-rə} and the past evidential \ipa{-sə}.

The testimonial\footnote{On this term see \citealt{tournadre08conjunct}; it corresponds to what \citealt{aikhenvald06} calls `sensory'.} \ipa{-rə} is generally used to express a non-past state or an action that the speaker is directly witnessing, be it by vision or by other senses. Thus, sentence \ref{ex:NErE} can be uttered by someone who sees (or feels, in a car) the state of the road.

 \begin{exe}
\ex \label{ex:NErE}
\gll
\ipa{tɕe} 	\ipa{ke} 	\ipa{rcʚ} 	\ipa{gə} 	\ipa{ŋə-rə}  \\
road very bad \textsc{indef} be-\textsc{testim} \\
\glt It is a bad road.
 \end{exe}
 
The testimonial \ipa{-rə} is not used to refer to observable statements about the speaker himself: in \ref{ex:YJEraNkhE},  the verb \ipa{ŋə-ã} cannot take  the suffix \ipa{-rə}. It is not used either with third person referents when the speaker  considers the statement  to belong to generally accepted `encyclopaedic' knowledge.\footnote{This category is referred to as `factual' in   \citet{tournadre08conjunct}.} 

  \begin{exe}
\ex \label{ex:YJEraNkhE}
\gll
  \ipa{ŋa} 	\ipa{ke} \ipa{mbjoˠ} 	\ipa{ɲɟəra-ŋkʰə} 	\ipa{gə} 	\ipa{ŋə-ã} \\
\textsc{1sg} very be.fast run-\textsc{nmlz}:S/A \textsc{indef} be-1 \\
\glt I am a fast runner.
  \end{exe}
  
 
  
As the testimonial \ipa{ɴdug} in Tibetan, \ipa{-rə}  can be used on the other hand   to express endopathic sensations, knowledge (\ref{ex:XamigoN}) or desire (\ref{ex:prerE}) with the first person.

\begin{exe}
\ex \label{ex:XamigoN}
\gll
\ipa{ŋa} 	\ipa{sə} 	\ipa{rdʑi-ʁa} 	\ipa{tɕʰu} 	\ipa{χa} 	\ipa{mi-gʚ-ã-rə} 	\\
\textsc{1sg} \textsc{top} trace-\textsc{loc} anything understand(1) \textsc{neg}-understand(2)-1-\textsc{testim} \\
\glt Me, on the other hand, I do not understand anything about tracking (animals) (The hybrid yak, 17)
\end{exe}

\begin{exe}
\ex \label{ex:prerE}
\gll
\ipa{ŋa} 	\ipa{tɕa} 	\ipa{tʰi-sɲə} 	\ipa{pre-rə} \\
\textsc{1sg} tea drink-want want-\textsc{testim} \\
\glt I want to drink tea.
\end{exe}

The perfective evidential \ipa{-sə} indicates that the speaker learnt of the facts in question second-hand (hearsay) or guessed them from indirect evidence (inferential), as in sentence \ref{ex:nEfse2}.

\begin{exe}
\ex \label{ex:nEfse2}
\gll \ipa{tʂaɕi-w} 	\ipa{dʐʚma} 	\ipa{de} 	\ipa{nə-f-se-sə}  \\
Bkrashis-\textsc{erg} Sgrolma \textsc{dem} \textsc{pfv-inv}-kill-\textsc{evd} \\
\glt Bkrashis killed Sgrolma.
\end{exe}

A perfective sentence without the suffix  \ipa{-sə} is used if the speaker has first-hand authoritative knowledge on the events described. For instance, \ref{ex:nEfse} can only be said by someone who witnessed the crime.


\begin{exe}
\ex \label{ex:nEfse}
\gll \ipa{tʂaɕi-w} 	\ipa{dʐʚma} 	\ipa{de} 	\ipa{nə-f-se}  \\
Bkrashis-\textsc{erg} Sgrolma \textsc{dem} \textsc{pfv-inv}-kill \\
\glt Bkrashis killed Sgrolma.
\end{exe}

 

A verb in the perfective evidential can be combined with the testimonial form of the verb \ipa{ŋə} `be' to form the narrative compound tense (see \ref{ex:nEfse3}), which is specifically used to tell stories. It is the most common verb form in narratives and other traditional stories.

\begin{exe}
\ex \label{ex:nEfse3}
\gll \ipa{tʂaɕi-w} 	\ipa{dʐʚma} 	\ipa{de} 	\ipa{nə-f-se-sə}  \ipa{ŋə-rə} \\
Bkrashis-\textsc{erg} Sgrolma \textsc{dem} \textsc{pfv-inv}-kill-\textsc{evd} be-\textsc{testim} \\
\glt Bkrashis killed Sgrolma.
\end{exe}
 

 
 
 
\section{Classification} \label{sec:classification}
 There is clear evidence from morphology and lexicon that Stau and Khroskyabs languages constitute a subgroup within Rgyalrongic.
 
 First, both Stau and Khroskyabs languages have generalized the inverse forms in the non-local scenario and completely lost the direct 3$\rightarrow$3' forms (\citealt{lai13affixale}, \citealt{jacques14inverse},   \citealt{lai14person}), a puzzling feature that is unlikely to be an independent innovation, as there is no comparable example in any other language family.
 
 Second, Stau and Khroskyabs lost the nominalization prefixes found in Rgyalrong languages. Only indirect traces of the prefixes remain.
 
 An example of such a trace is the noun \ipa{ɣɟi}  `hole, orifice' in Khanggsar Stau,  \ipa{ʁɟô} in Wobzi Khroskyabs, both   cognate to Japhug \ipa{--ɣɲɟɯ}  `(its) hole, opening', which  derives from \ipa{ɲɟɯ} `be opened (intr)', the anticausative of \ipa{cɯ} `open (a door)'. 
 
 In Japhug, \ipa{ɣ--} is an irregular allomorph of the nominalization prefix \ipa{kɯ--}, found in a handful of examples (\citealt[4-6]{jacques14antipassive}). Although the \ipa{ɣ--} element in Stau and \ipa{ʁ--} in Khroskyabs are not analyzable anymore (the forms \ipa{ɣɟi}  and   \ipa{ʁɟô} are synchronically unmotivated), this example and others (see \citealt[1228-9]{jacques12incorp}) suggest  that nominalization prefixes used to exist in Stau and Khroskyabs, and were later replaced by innovative suffixes.  
 
 These suffixes, probably originally generic relator nouns, are partially shared between Stau and Khroskyabs (see Table \ref{tab:nmlz}), in particular the S/A nominalizer \ipa{-ŋkʰə} and the oblique nominalizer \ipa{-re}. Although contact cannot be excluded as a factor in the development of these suffixal systems, no discussion of Rgyalrongic subgrouping can neglect these data. 
 
 The suffix \ipa{-ŋkʰə}  in Stau and Khroskyabs is reminiscent of Tibetan \ipa{--mkʰan} which has the same use in some modern languages, but this is likely a coincidence (the rhyme \ipa{--an} never corresponds to \ipa{-ə} in loanwords).
 
 
 
 
  \begin{table}[h]
  \caption{Nominalization suffixes in Stau and Khroskyabs} \label{tab:nmlz} \centering
 \begin{tabular}{llllll}
 \toprule
 &	Stau &	Khroskyabs &	\\	
 \midrule
S/A &	\ipa{-ŋkʰə} &	\ipa{-pa, -ŋkʰə} &	\\
P, action &	\ipa{-lə} &	\ipa{-spi} &	\\
locative &	\ipa{-re} &	\ipa{-ri} &	\\
instrument &	\ipa{-sce} &	\ipa{-ri} &	\\
\bottomrule
\end{tabular}
\end{table}

Third, both Stau and Khroskyabs present a type of verbal reduplication unattested elsewhere, whereby the replicated syllable appears after the base, and its rhyme is replaced by \ipa{--a} (\citealt{lai13fuyin}). This process is productive is Stau and Khroskyabs (for instance \ipa{ŋgə} `eat' $\rightarrow$ \ipa{ŋgəŋga} `eat all kinds of things').
 
 
 Fourth, Stau and Khroskyabs languages share several lexical isoglosses which distinguish them from Rgyalrong languages, as illustrated by the data in Table \ref{tab:lexicon}. In all these examples Khroskyabs and Stau differ from Japhug and it can be convincingly shown in some cases that they have innovated. In the case of nouns like \ipa{mkʰə}  smoke', the innovation is the replacement of the simple root by a compound comprising the original root and another element (fire+smoke $\rightarrow$ smoke, a well-attested unidirectional semantic change, see \citealt{urban11semantic}).

  \begin{table}[h]
  \caption{Potential lexical innovations} \label{tab:lexicon} \centering
 \begin{tabular}{llllll}
 \toprule
 &	Stau &	Khroskyabs &	Japhug &	\\	
  \midrule
heart &	\ipa{zjar}  &	\ipa{sjɑ̂r} &	\ipa{tɯ-sni} &	\\	
smoke  &	\ipa{mkʰə} &	\ipa{mkʰə́} &	\ipa{tɤ-kʰɯ} &	\\	
be big  &	\ipa{cʰe} &	\ipa{cʰæ̂} &	\ipa{wxti} &	\\	
bread  &	\ipa{ləkʰi} &	\ipa{lækʰí} &	\ipa{qɤjɣi} &	\\	
writing  &	\ipa{tɕədə} &	\ipa{dʑədə́} &	\ipa{tɤscoz} &	\\	
wind  &	\ipa{χpərju} &	\ipa{χpə̂rju    } &	\ipa{qale} &	\\	
experience  &	\ipa{zdar} &	\ipa{zdɑ̂r} &	\ipa{rɲo} &	\\	
skin  &	\ipa{tɕədʑa} &	\ipa{dʑədʑɑ̂} &	\ipa{tɯ-ndʐi} &	\\	
water  &	\ipa{ɣrə} &	\ipa{jdə̂} &	\ipa{tɯ-ci} &	\\
general classifier  &	\ipa{ə-lʚ} &	\ipa{ə̂-lo} &	\ipa{tɯ-rdoʁ} &	\\	
human classifier  &	\ipa{a-ʁi} &	\ipa{ə̂-ʁæi} &	\ipa{tɯ-rdoʁ} &	\\	
\bottomrule
\end{tabular}
\end{table}

Stau and Khroskyabs are also characterized by a series of retentions not shared by Rgyalrong languages. Although these do not provide evidence for the Stau-Khroskyabs branch, they are nevertheless worth mentioning.

First, Stau and Khroskyabs have two distinct roots for `year', one with the numeral prefixes (\ipa{--fku}) and the other in year ordinals (\ipa{--və}). Wobzi Khroskyabs \ipa{--dju} corresponds to Thugsrjechenbo \ipa{--dɣu} and is related to Stau \ipa{--fku} (the velar stop underwent lenition in Khroskyabs languages). Japhug and other Rgyalrong languages, on the other hand, have the root \ipa{--xpa} / \ipa{--pa} everywhere. 

A root suppletion in the noun `year' cognate with the one in Khroskyabs and Stau is found in Lolo-Burmese and Naish (see \citealt{jacques.michaud11naish}), showing that Rgyalrong languages are innovative here.

  \begin{table}[h]
  \caption{Year ordinals in Rgyalrongic languages} \label{tab:year} \centering
 \begin{tabular}{llllll}
 \toprule
 & 	Stau & 	Khroskyabs & 	Japhug & 	\\	
 \midrule
one year & 	\ipa{e-fku} & 	\ipa{ə̂-dju } & 	\ipa{tɯ-xpa} & 	\\	
two years & 	\ipa{ɣnə-fku} & 	\ipa{jnæ̂-dju } & 	\ipa{ʁnɯ-xpa} & 	\\	
three years & 	\ipa{xsʚ-fku} & 	\ipa{çsô-dju } & 	\ipa{χsɯ-xpa} & 	\\	
 \midrule
last year & 	\ipa{javə} & 	\ipa{aχpî} & 	\ipa{japa} & 	\\	
this year & 	\ipa{pəvə} & 	\ipa{pîvi} & 	\ipa{ɣɯjpa} & 	\\	
%next year & 	\ipa{sevə} & 	\ipa{fsǽpi} & 	\ipa{fsaqhe} & 	\\	
\bottomrule
\end{tabular}
\end{table}


Second, Khroskyabs and Stau preserve the prohibitive dental prefix (\ipa{tə--} in Wobzi \citealt[130-1]{lai13affixale}, \ipa{di--} in Stau) that does not exist in Rgyalrong languages (which instead have a prohibitive form \ipa{ma--}).

Third, directional prefixes are placed \textit{before} the negation in Stau and Khroskyabs, while they appear after it in Rgyalrong languages. Since the order negation-directional is also found in related languages such as Tangut and Pumi (\citealt{jacques11tangut.verb}), Rgyalrong languages are most probably innovative here.

\bibliographystyle{linquiry2}
\bibliography{bibliogj}
\end{document}
