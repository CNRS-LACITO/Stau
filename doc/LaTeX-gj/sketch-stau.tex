\documentclass[oneside,a4paper,11pt]{article} 
\usepackage{fontspec}
\usepackage{natbib}
\usepackage{booktabs}
\usepackage{xltxtra} 
\usepackage{longtable}
\usepackage{polyglossia}
\usepackage[top=72pt,bottom=72pt,left=72pt,right=72pt]{geometry}
\usepackage[table]{xcolor}
\usepackage{gb4e} 
\usepackage{multicol,multirow}
\usepackage{graphicx}
\usepackage{float}
\usepackage{slashbox} 
\usepackage{rotating}
\usepackage{hyperref} 
\hypersetup{bookmarksnumbered,bookmarksopenlevel=5,bookmarksdepth=5,colorlinks=true,linkcolor=blue,citecolor=blue}
\usepackage[all]{hypcap}
\usepackage{memhfixc}
\usepackage{lscape}
\bibpunct[: ]{(}{)}{,}{a}{}{,}
%%%%%%%%%quelques options de style%%%%%%%%
%\setsecheadstyle{\SingleSpacing\LARGE\scshape\raggedright\MakeLowercase}
%\setsubsecheadstyle{\SingleSpacing\Large\itshape\raggedright}
%\setsubsubsecheadstyle{\SingleSpacing\itshape\raggedright}
%\chapterstyle{veelo}
%\setsecnumdepth{subsubsection}
%%%%%%%%%%%%%%%%%%%%%%%%%%%%%%%
\setmainfont[Mapping=tex-text,Numbers=OldStyle,Ligatures=Common]{Charis SIL} %ici on définit la police par défaut du texte
\renewcommand \thesection {\arabic{section}.}
\renewcommand \thesubsection {\arabic{section}.\arabic{subsection}.}
\newfontfamily\phon[Mapping=tex-text,Ligatures=Common,Scale=MatchLowercase,FakeSlant=0.3]{Charis SIL} 
\newcommand{\ipa}[1]{{\phon #1}} %API tjs en italique
 
\newcommand{\grise}[1]{\cellcolor{lightgray}\textbf{#1}}
\newfontfamily\cn[Mapping=tex-text,Scale=MatchUppercase]{MingLiU}%pour le chinois
\newcommand{\zh}[1]{{\cn #1}}

\newcommand{\jg}[1]{\ipa{#1}\index{Japhug #1}}
\newcommand{\wav}[1]{#1.wav}
\newcommand{\tgz}[1]{\mo{#1} \tg{#1}}

\XeTeXlinebreaklocale 'zh' %使用中文换行
\XeTeXlinebreakskip = 0pt plus 1pt %
 \newcommand{\ra}{$\Sigma_1$} 
\newcommand{\rc}{$\Sigma_3$} 
\newcommand{\ro}{$\Sigma$} 

 
 
\begin{document} 


\title{Stau}
\author{Guillaume Jacques, Anton Antonov, Lai Yunfan and Lobsang Nima}

\maketitle

 

\section{Introduction}
The cluster of languages variously referred to as Stau, Ergong or Horpa in the literature are spoken over a large area from Ndzamthang county (in Chinese Rangtang  \zh{壤塘县})  in Rngaba  prefecture (Aba \zh{阿坝州}) to Rtau county (Daofu \zh{道孚}) in Dkarmdzes prefecture (Ganzi \zh{甘孜州}), in sichuan province, China. At the moment of writing, it is still unclear how many unintelligible varieties belong to this group, but at least three must be distinguished: the language of Rtau county (referred as `Stau' in this paper), the Dgebshes language (Geshizha \zh{格什扎话})  spoken in Rongbrag county (Danba \zh{丹巴}), and the Stodsde language (Shangzhai \zh{上寨}) in Ndzamthang.  

Research on these languages is still limited. No dictionaries or text corpora have yet been published. Stodsde is only known by a few articles (\citealt{jackson00sidaba} and   \citealt{jackson07shangzhai}), Dgebshes by a grammatical sketch (\citealt{duoerji98geshizha}). The dialects spoken in Rtau county have been investigated by several teams of scholars (\citealt{huangbf91daofu}, \citealt{sun13gexi} and \citealt{jacques14rtau}), but no detailed description of this language is available.

There is no consensus as to how these languages should be named. The present paper favours the Tibetan names of these languages rather than the Chinese-based ones, since Chinese names are transcription of the Tibetan names. 

  The native speaker among the authors (Lobsang Nima) favours the name \ipa{rəsɲəske} for his language (the Khang.gsar \ipa{qʰərŋe} dialect in Rtau county, in Chinese \zh{孔色} Kongse), but this name is not used by all speakers and we prefer the geographically-based name  `Stau'. The spelling with \textit{St--} rather than \textit{Rt--}, aside from being more pronounceable for the average western reader, better reflects the local pronunciation of the county name \ipa{stʚwu}. This spelling has already been used in English (see \citealt{wang70stau}).   
  
From the group comprising all three languages (Stau, Dgebshes and Stodsde),  we can follow 
\citet{jackson00sidaba}'s term `Horpa' which has indeed been used for this area in the past, or the related `Tre-Hor'.  The term Ergong \zh{尔龚} used by scholars such as \citet{sun83liujiang}, on the other hand, appears to lack any basis in the local languages and should be avoided.
 
As shown in section \ref{sec:classification}, there is evidence from verbal morphology and lexicon that Horpa languages form a subgroup within Rgyalrongic with Khroskyabs (previously known by the incorrect term `Lavrung'), as these two branches present common innovations that are unlikely to be explainable as parallel developments.
 
\section{Phonology}
 
 
  \subsection{Onsets}
  Table \ref{tab:consonants} presents the consonantal inventory postulated for Stau. Unlike in Rgyalrong languages, there is no evidence for treating the prenasalized voiced stops as single phonemes in Stau.
  
 \begin{table}
 \caption{Consonantal phonemes in Stau} \label{tab:consonants}  \centering
 \resizebox{\columnwidth}{!}{
\begin{tabular}{llllllllll}
\toprule
& &Bilabial& Labiodental &Dental/Alveolar & Retroflex &Alveolo-palatal&Palatal & Velar&Uvular\\
\midrule
Stop&unvoiced & p & & t & & & c & k & q\\
&aspirated & pʰ  && tʰ & & & cʰ & kʰ & qʰ\\
&voiced & b & & d & & & ɟ & g &  \\
Affricate&unvoiced &  &  & ts & tʂ& tɕ &  &   &  \\
&aspirated &   && tsʰ & tʂʰ& tɕʰ &  &   &  \\
&voiced &   &  &dz & dʐ& dʑ &  &   &  \\
Nasal &&m& & n   &&&ɲ &ŋ \\
Fricative &unvoiced&  &f & s & ʂ& ɕ &  &x   &χ  \\
&voiced&&v & z & & ʑ &  &ɣ   &ʁ  \\
Approximant && w &&&  &&j &\\
Rhotic &&&&&r&&\\
Lateral &sonorant &&&l\\
&  fricative &&&ɬ\\
\bottomrule
\end{tabular}}
\end{table}

Some dialects of Stau have contrastive aspirated fricatives (see \citealt{jackson00puxi},  \citealt{jacques11lingua}). In the Khang.gsar dialect, the unvoiced fricative phonemes are realized as aspirated in syllable-initial position without a cluster, and as unaspirated when a cluster is present. Thus, \ipa{nə-sow} `I killed him' is realized as [nəsʰow], while no such as aspiration is observed in \ipa{nə-fse} `he killed him' due to the cluster.

Voiced stops are almost absent word-initially in the Khang.gsar dialect of Stau. In the case of verbs, the voiced stop of affricate resurfaces when a prefix is added, as in the examples in Table \ref{tab:voiced} (the bare stem appears in non-finite and in factual forms). In word-initial clusters whose second element is a voiced stop, voicing is preserved, as in \ipa{zgə} `cause to wear, put clothes on', a verb derived from \ipa{kə} `wear'. 

 \begin{table}[H]
 \caption{Neutralization of voiced stops in word-initial position} \label{tab:voiced} \centering 
\begin{tabular}{lllll}
\toprule
Bare stem & Meaning & Prefixed verb & Meaning \\
\midrule
\ipa{pərje} & burn (it) &\ipa{tə-bərje-sə} & it burnt \\
\ipa{te} & do &\ipa{nə-dej} & do it !\\
\ipa{tɕə} & meet &\ipa{kə-dʑõ} & I met (him)\\
\ipa{kə} &wear &\ipa{rə-gu} & I wore it \\
\bottomrule
\end{tabular}
\end{table}
Unvoiced stops and affricative do not have any alternation, as for instance \ipa{tɕi} `wear (hat)' $\rightarrow$ \ipa{rə-tɕu} `I wore it'.

Nouns do not have any productive prefix, and thus the contrast has been lost, except for   \ipa{bəlʚ} `cheek', the only native word whose voicing is preserved word-initially.

 
The status of the unvoiced labiodental fricative \ipa{f} is unclear. This sound is never found as a simple onset in native words (only in borrowings from Chinese). However, some clusters such as  \ipa{ʂf} [r̥ɸʷ] (contrasting with \ipa{rv}), where \ipa{f} is the element closest to the vowel, are difficult to account for without postulating an unvoiced labial fricative phoneme.
 
 In clusters with a nasal as a first element, only stops and affricates are attested, except for the cluster \ipa{nɬ}, which is in free variation with \ipa{ntʰ}.
 
% \ipa{ɦwa} `hug'
  \subsection{Rhymes}
 
There are ten vowels in the Khang.gsar dialect of Stau, six plain vowels (--\ipa{i}, --\ipa{e}, --\ipa{a}, --\ipa{ə}, --\ipa{ʚ},  --\ipa{u}), two velarized vowels (--\ipa{oˠ}, --\ipa{aˠ}) and two nasal vowels (--\ipa{õ}   --\ipa{ã}).

The velarized vowels are almost exclusively attested in Tibetan loanword (--\ipa{oˠ} corresponds to Tibetan \ipa{--og} and --\ipa{aˠ} to \ipa{--ag} and \ipa{--eg}), but there are a few native words with velarized vowels too, such as \ipa{mbjoˠ} `fast' (cognate to Japhug \ipa{mbjom} `fast').

Only three codas are possible in Stau: \ipa{--v},  \ipa{--r} and  \ipa{--m}, the latter only attested in Tibetan loanwords. In native words, the coda \ipa{--r} sometimes appear to correspond  to \ipa{--r} in other Rgyalrongic languages (as in \ipa{χtɕʰər} `sour', Japhug \ipa{tɕur}), but in other cases, such as \ipa{spar} `be thursty' (Japhug \ipa{ɕpaʁ}) or \ipa{zdʚr}  cloud, be cloud' (Japhug \ipa{zdɯm} `cloud'), there is no \ipa{--r} coda anywhere else in the family. A possible explanation is that the testimonial suffix \ipa{--rə}, whose vowel tend to be elided, has been reanalyzed as part of the stem in third person forms (see section \ref{sec:TAM}).

 \subsection{Vowel fusion}
 In order to account for vowel alternations observed in the verbal and nominal systems, \citet{jacques14rtau} postulate a series of vowel fusion rules, summarized in \ref{tab:alternation} (the symbol C stands for either final \ipa{--r} or final \ipa{--v}).
\begin{table}[H]
\caption{Vowel fusion in Stau} \label{tab:alternation} \centering
\begin{tabular}{c|cccccc}
\toprule

 \backslashbox{Stem}{Suffix} &  	--\ipa{w} &  --\ipa{ã} &  --\ipa{j} \\
\hline
\ipa{i}&\ipa{u}&\ipa{ã}&\ipa{i}\\
\ipa{e}&\ipa{ow}&\ipa{ã}&\ipa{ej}\\
\ipa{a}&\ipa{ow}&\ipa{ã}&\ipa{ej}\\
\ipa{ə}&\ipa{u}&\ipa{õ}&\ipa{i}\\
\ipa{ʚ}&\ipa{ow}&\ipa{õ}&\ipa{ej}\\
\midrule
\ipa{əv}&\ipa{u}&\ipa{õ}&\ipa{iv}\\
\ipa{ər}&\ipa{u}&\ipa{õ}&\ipa{i}\\
\ipa{a}C&\ipa{ow}&\ipa{ã}&\ipa{ej}\\
\ipa{e}C&\ipa{ow}&\ipa{ã}&\ipa{ej}\\
\ipa{ʚ}C &\ipa{ow}&\ipa{õ}&\ipa{ej}\\
\bottomrule
\end{tabular}
\end{table}


Vowel fusion cannot occur with the vowels -\ipa{u}, --\ipa{oˠ}, --\ipa{aˠ}, --\ipa{õ} and --\ipa{ã} and the rhymes ending in \ipa{--m}.
 
\section{Verbal morphology}


\subsection{Intransitive conjugation}
In the intransitive conjugation, the third and second persons singular forms are in the bare stem, while first person (singular and plural) forms have a suffix \ipa{--ã}.

Following the rules of  vowel fusion in Table \ref{tab:alternation}, vowel fusion between the first person suffix \ipa{--ã} and the verb stem has differ results depending on the vowel (and coda) of the verb stem. 
 
Six classes of alternations are found in verbs with open syllables; class 6 includes verbs without alternation, whose rhyme can be any of --\ipa{u}, --\ipa{oˠ}, --\ipa{aˠ}, --\ipa{õ} and --\ipa{ã}:
\begin{table}[H]
\caption{Vowel alternations in open-syllable intransitive verbs in Rtau} \label{tab:open.intr} \centering
\begin{tabular}{llll|ll|l}
\toprule
%Verb class&1&2&3&4&5&6 \\
Meaning &	look at   &  	move   &  	like&  	be full     &  	 	be ill      &  	be hot       \\  
\midrule
1&	\ipa{scəqã} & 	\ipa{mbəɕã} & \ipa{rgã} &	\ipa{fkõ} & 	  	\ipa{ŋõ} & 	   	\ipa{cʰu}   \\ 
1 (underlying)&	\ipa{scəqi-ã} & 	\ipa{mbəɕe-ã} & \ipa{rga-ã} &	\ipa{fkə-ã} & 	  	\ipa{ŋɞ-ã} & 	   	\ipa{cʰu-ã}   \\ 
2/3&	\ipa{scəqi} & 	\ipa{mbəɕe} & \ipa{rga} & 	\ipa{fkə} & 	  	\ipa{ŋɞ} & 	 	\ipa{cʰu}  \\ 
\bottomrule
\end{tabular}
\end{table}


There are two irregular verbs with intransitive morphology in Japhug with have \ipa{--ã} in the first person instead of expected \ipa{--õ} (cf table \ref{tab:irr.intr}). 

\begin{table}[H]
\caption{Irregular intransitive verbs in Rtau} \label{tab:irr.intr} \centering
\begin{tabular}{llll|ll|l}
\toprule
meaning &	go     & say \\  
\midrule
1&	\ipa{ɕã}  	 &\ipa{jã}\\ 
2/3&	\ipa{ɕə} & 	\ipa{jə} &\\ 
\bottomrule
\end{tabular}
\end{table}

 \subsection{Transitive conjugation}
The basic structure of transitive conjugations  in Japhug can be illustrated by Table \ref{tab:kill} (the columns represent the P and the lines the A). In addition to the first person suffix \ipa{--ã} , we find two additional suffix (\textsc{1sg}$\rightarrow$3  \ipa{--w} and 2$\rightarrow$3 \ipa{--j}) and the inverse prefix \ipa{f--/v--}. The only unmarked form in the paradigm is the 1$\rightarrow$2 slot, a curious fact which however can be accounted for historically (see \citealt{jacques14rtau}).



\begin{table}[h]
\caption{\ipa{fse} `kill'}
\centering \label{tab:kill}
\begin{tabular}{|c|cc|c|c|}  
 \cline{1-5}
\backslashbox{A}{P} &\textsc{1sg}  &  \textsc{1pl}  &  2  &  	3  \\  
\cline{1-5}
 \textsc{1sg}  &  	 \multicolumn{2}{c}{\cellcolor{lightgray}}   \vline    &  	\multirow{2}{*}{\ipa{--se}}  &  	\ipa{--se-w}  \\  
\cline{5-5}\textsc{1pl}  &  \multicolumn{2}{c}{\cellcolor{lightgray}} 	 \vline   &   &  	\ipa{--se-ã}  \\  
\cline{5-5}2 &    \multicolumn{2}{c}{\multirow{2}{*}{\ipa{--f-se-ã}}}    \vline  &   \grise{ }	  &  	\ipa{--se-j}  \\  
\cline{5-5}3 &  \multicolumn{2}{c}{ } \vline &  	\multicolumn{2}{c}{ \ipa{--f-se}}   	 \vline  \\  
\cline{1-5}
\end{tabular}
\end{table}

The vowel fusion rules in Table \ref{tab:alternation} apply to all suffixed forms, as in the intransitive paradigms. No example of irregular vowel fusion has yet been discovered with transitive verbs.

The \ipa{f}-- / \ipa{v}-- prefix appears in 2/3$\rightarrow$1, 3$\rightarrow$2 and 3$\rightarrow$3 forms. Its presence  in 2$\rightarrow$1 precludes an analysis as a third person agent marker, and it is best to treat it as an inverse marker, although there is no obviation contrast in  3$\rightarrow$3 forms (see section \ref{sec:classification}). The inverse prefix does not occur with verb stems containing an initial cluster, or with monosyllabic verbs with the coda \ipa{--v} such as \ipa{kʰev} `scoop (water)'. 



 
%
%
%
%The inverse \ipa{v}-- prefix appears in all 3$\rightarrow$3 forms in Rtau, a feature shared with Khroskyabs. Both Rtau and Khroskyabs differ from Rgyalrong languages, where two 3$\rightarrow$3 forms are found: the \textit{direct}   and the \textit{inverse} form. 
%The inverse \ipa{v}-- prefix presents phonological alternations and phonotactic constraints. It is prefixed to the first syllable of the verb stem, even when polysyllabic. In verbs with reduplicated stem, such a `wipe' (Table \ref{tab:wipe}, \ipa{nə--} here is the directional prefix, see section \ref{sec:dir.pref}), reduplication also applies to the inverse prefix.
%
%  \begin{table}[H]
%\caption{\ipa{f-ɕə-f-ɕe} `wipe'}
%\centering \label{tab:wipe}
%\begin{tabular}{|c|c|c|c|c|}  
% \cline{1-4}
%\backslashbox{A}{P} &1    &  2  &  	3  \\  
%\cline{1-4} 1s  &   \cellcolor{lightgray}        &  	\multirow{2}{*}{\ipa{nə-ɕəɕe}}  &  	\ipa{nə-ɕəɕo-w}  \\  
%\cline{4-4}1p  &   \cellcolor{lightgray} 	     &   &  	\ipa{nə-ɕəɕ-ã}  \\  
%%\hline
%\cline{4-4}2 &   \multirow{2}{*}{\ipa{nə-fɕəfɕ-ã}}     &   \grise{ }	  &  	\ipa{nə-ɕəɕe-j}  \\  
%\cline{3-4}3 &    &  	\multicolumn{2}{c}{ \ipa{nə-fɕəfɕe}}   	 \vline  \\  
%\hline
%\end{tabular}
%\end{table}
%
%The \ipa{v}-- prefix is assimilated  to \ipa{f}-- when prefixed to a verb stem with unvoiced initial consonant (as in \ipa{f-se} [\textsc{inv}-kill] `he kills'). It cannot be inserted whenever any of the following three conditions apply:
%
%\begin{itemize}
%\item When the stem-initial consonant is a labial (either /\ipa{p}/, /\ipa{b}/, /\ipa{m}/, or  /\ipa{v}/) or the voiced uvular /\ipa{ʁ}/, the inverse cannot be prefixed. Thus the third person form of \ipa{və} `do' \ipa{ʁʚ} `help' are identical to the corresponding bare stems.
%\item The inverse prefix is not compatible with most stem-initial clusters. The only clusters that allow prefixation of \ipa{v}-- are /stop+r/ clusters. For instance, the root /\ipa{kʰrə}/ `hold' (\textsc{1sg$\rightarrow$3} \ipa{kʰru}) thus has a 3$\rightarrow$3 form \ipa{f-kʰrə}, whereas \ipa{zjə} `sell' has a third person form identical to the bare stem (the cluster *\ipa{vzj}-- is not allowed in the variety of Rtau under study).
%\item The inverse does not appear in transitive verbs with final \ipa{--v}, due to a dissimilatory constraint. For instance, the 3$\rightarrow$3 forms of /\ipa{kʰev}/ `scoop' and /\ipa{ɕev}/ `take out' are \ipa{\ipa{kʰev}} and \ipa{\ipa{ɕev}} respectively, not *\ipa{fkʰev} or *\ipa{fɕev}.
%\end{itemize}
%
 
%
%The morphologically based distinction between transitive and intransitive verbs in Rtau must be refined by taking into account case-marking on arguments. 
%
%Rtau, as all Rgyalrongic languages, is a strict verb-final language with postpositions. Case markers include the ergative \ipa{--w}, the genitive \ipa{--j}, the dative \ipa{--gi} and the instrumental \ipa{--kʰa}. Only animate referents can receive   ergative marking, inanimates can only be marked with the instrumental. SAP pronouns are not normally marked with the ergative (except in some subordinate clauses).
%
%
%Some verbs with intransitive morphology, such as `like', do require ergative marking on the argument whose person is indexed on the verb, as illustrated by examples \ref{ex:rgaN} and \ref{ex:rga}.
%\begin{exe}
%\ex \label{ex:rgaN}
%\gll \ipa{ŋa}  	\ipa{tə-ɡi}  	\ipa{rɡa-ã-rə}  \\
%I he-\textsc{dat} like-1-\textsc{const} \\
%\glt `I like him/her.'
%\end{exe}
%
%\begin{exe}
%\ex \label{ex:rga}
%\gll \ipa{tə-w}  	\ipa{ŋa-ɡi}  	\ipa{rɡa-rə}  
% \\
%he-\textsc{erg} I-\textsc{dat} like-\textsc{const} \\
%\glt `(S)he likes me.'
%\end{exe}
%
%This type of semi-transitive verb (which are transitive from the point of view of case marking and intransitive from that of verb morphology) include some experiencer verbs like \ipa{rga} `like' and some speech verbs like \ipa{jə} `say'. The stimulus or the addressee is marked with the dative case.
%
%Some verbs with transitive morphology agree with only one of their arguments. Thus, /si/ `know (somebody)' indexes  the person knowing, while the P is always third person by default, as shown in Table \ref{tab:know}.
%
%\begin{table}[H]
%\caption{\ipa{f-si} `know'}
%\centering \label{tab:know}
%\begin{tabular}{|c|c|c|c|c|}  
% \cline{1-4}
%\backslashbox{A}{P} &1    &  2  &  	3  \\  
%\cline{1-4} 1s  &   \cellcolor{lightgray}        &  	\multicolumn{2}{c}{\ipa{su}}  \vline  \\  
%\cline{3-4}1p  &   \cellcolor{lightgray} 	     &  \multicolumn{2}{c}{\ipa{sã}}\vline  \\  
%%\hline
%\cline{4-4}2 &    \ipa{si}     &   \grise{ }	  &  	 \ipa{si}  \\  
%\cline{2-4}3 &     	\multicolumn{3}{c}{ \ipa{fsi}}   	 \vline  \\  
%\hline
%\end{tabular}
%\end{table}
%
%When the person known is an SAP, an overt  pronoun must be used, and appears in the absolutive form (example \ref{ex:fsi}).
%
%\begin{exe}
%\ex \label{ex:fsi}
%\gll \ipa{tə-w}  	\ipa{ŋa}  	\ipa{f-si}  
% \\
%he-\textsc{erg} I \textsc{inv}-know \\
%\glt `S/he knows me'.
%\end{exe}
%
%
%Ditransitive verbs that index the recipient as the P (\textit{secundative} in \citealt{malchukov10ditransitive}'s terminology), the recipient still receives dative marking despites being indexed in the verb morphology, as in example \ref{ex:xsev} with the verb /xsev/ `give back'.
%
%\begin{exe}
%\ex \label{ex:xsev}
%\glt \ipa{təɲu ŋaɲəgi kəxsã}
%\gll
%\ipa{tə-ɲə-w}  	\ipa{ŋa-ɲə-gi}  	\ipa{kə-v-xsev-ã.}  \\
%3-\textsc{pl-erg} 1-\textsc{pl-dat} \textsc{pfv-inv}-return-1 \\
%\glt They gave it back to us.
%\end{exe}
%
 
%As in all Rgyalrongic languages, Rtau has a system of five directional prefixes used to indicate both direction and TAM. The prefixes come in two series, one used for perfective and imperative forms (with \ipa{ə} vocalism), and another one for perfective interrogative (with \ipa{i} vocalism and stress), as indicated in Table \ref{tab:dir.pref}.
%
%
%
%\begin{table}[H]
%\caption{Directional prefixes in Rtau} \label{tab:dir.pref} \centering
%\begin{tabular}{lccccc}
%\toprule
%Direction & Perfective / Imperative & Interrogative \\
%\midrule
% Up & \ipa{rə--} & \ipa{rí--} \\
%Down & \ipa{nə--} & \ipa{ní--} \\
%North  & \ipa{kə--} & \ipa{kí--} \\
%South & \ipa{ɣə--} & \ipa{ɣí--} \\
%No direction & \ipa{tə--} & \ipa{tí--} \\
%\bottomrule
%\end{tabular}
%\end{table}
%
%The prefixes \ipa{kə--} and \ipa{ɣə--} are here glossed as `north' and `south' rather than `translocative' (\zh{离心}) and  `cislocative' (\zh{向心}) as in \citet[26]{huangbf91daofu}. At least in the variety under study, the  use of these two prefixes is not determined by the relative direction towards or away from the main referent. For instance, in example \ref{ex:kE},   the prefix \ipa{kə--} appears with  the verb \ipa{ʂɸa} `come out, appear' (which is compatible with all directional prefixes) to express motion towards the main referent.
%
%\begin{exe}
%\ex \label{ex:kE}
%\gll 	\ipa{tʰaˠdʑi}  	\ipa{tʰaˠdʑi-jəkʰa}  	\ipa{raca}  	\ipa{kə-ʂɸa}  	\ipa{vdʚ-sə}  	\ipa{ŋə-rə}  \\
%far far-from horseman \textsc{pfv:north}-come.out see-\textsc{evd} be-\textsc{testim} \\
%\glt (Akhu stonba) saw  a horseman coming from afar (towards him from the south to the north). (Akhustonba and the horseman, 4)
%\end{exe}
%
%Verbs other than motion verbs generally  only allow one direction, which is lexically determined. Thus for instance \ipa{tʰi} `drink' appears with \ipa{ɣə--} `towards south' with \ipa{ŋgə} `eat' is used with \ipa{nə--} `down'.
%
%Unlike Rgyalrong languages and Khroskyabs, there is no regular stem alternation in Rtau related to TAM categories. However, there are two types of irregularities in TAM marking.
%
%First, a handful of verbs are never used with directional prefixes: this is the case of \ipa{vdʚ} `see' (example \ref{ex:kE} above; the evidential form in --\ipa{sə} normally requires a directional prefix), \ipa{ste} `finish', \ipa{si} `know'.
%
%Second, the motion verbs `come' and `go' are exceptional in that they allow directional prefixes in the non-past. The presence vs absence of directional prefixes is the only difference between perfective and non-past in most verbs, but in the case of \ipa{ɮde} `come' and \ipa{ɕə} `go' the suppletive stems  \ipa{--kʰi} and \ipa{--vi} respectively are used in the non-past with directional prefixes, as summarized in Table \ref{tab:motion.irr}.\footnote{There is in addition a defective motion verb \ipa{rja} `leave' only used in the third person perfective form; for the first and second person, corresponding forms of the verb \ipa{ɕə} must be used instead.}
%
%\begin{table}[H]
%\caption{Directional prefixes in Rtau} \label{tab:motion.irr} \centering
%\begin{tabular}{lccccc}
%\toprule
%Meaning & Perfective & Non-Past & Non-Past with directional prefixes \\
%\midrule
%go & \ipa{ɕə} &\ipa{ɕə} & \ipa{--vi} \\
%come & \ipa{ɮdi} &\ipa{ɮde} & \ipa{--kʰi} \\
%\bottomrule
%\end{tabular}
%\end{table}
%
%The verb `come' has distinct perfective and non-past stems. In the perfective \ipa{ɮdi} is most often used   without directional prefix (example \ref{ex:lZdi}), but using it with directional is nevertheless possible, unlike verbs such as \ipa{vdʚ} `see'.
%
%\begin{exe}
%\ex \label{ex:lZdi}
%\gll 
%\ipa{sa}  	\ipa{ʁjikʰoˠ}  	\ipa{rʚ}  	\ipa{ɮdi-sə}  	\ipa{ŋə-rə.}   \\
%place Gyukhog up come-\textsc{evd} be-\textsc{testim} \\
%\glt He came (up there) at the place (called) Gyukhog. (The thieves, 39)
%\end{exe}
%
%The non-past stem \ipa{ɮde} occurs in the non-past and imperative forms, and it is homophonous with the transitive verb |\ipa{ɮde}| `bring' (whose 3$\rightarrow$3 form is \ipa{vɮde} with the inverse prefix, thus never ambiguous with the intransitive verb).

 \subsection{Derivational morphology}
 
The only denominal prefix in Stau is \ipa{s--/z--}, cognate of Japhug \ipa{sɯ--/sɤ--/ɕɯ--} (on which see \citealt[14-17]{jacques14antipassive}). It derives transitive verbs, illustrated by the examples in Table \ref{tab:denominal}. While the original semantics of this prefix was likely `use X' and `cause sb. to have X' as in Japhug, in Stau the semantics of the denominal verbs are largely unpredictable, and result from semantic shifts (`use a staff' $\rightarrow$ `hit with a staff' $\rightarrow$  `hit'). The only denominal verb shared by Stau, Khroskyabs and Japhug is \ipa{smi} `give a name' (in Japhug \ipa{sɤrmi}).
 
 \begin{table}[H]
 \caption{Denominal verbs in Stau} \label{tab:denominal} \centering 
\begin{tabular}{lllll}
\toprule
Base noun & Meaning & Denominal verb & Meaning \\
\midrule
\ipa{pəcʰa} & staff&\ipa{zbəcʰa} & hit \\
\ipa{rmi} &name &\ipa{smi} &give a name \\
\ipa{ɣme} &wound &\ipa{smi} & hurt \\
\bottomrule
\end{tabular}
\end{table}
 
 
 Stau has a few example of anticausative verbs with voiced onset (Table \ref{tab:anticausative}). The anticausative verbs derive from the transitive ones (not the opposite direction of derivation, see \citealt{jacques12demotion}). There is an irregularity in the pair \ipa{ftɕə} vs \ipa{dʑə} `melt tr/it', as the \ipa{f--} element of the transitive verb has no equivalent in the anticausative one. Interestingly, the same irregularity is found in the cognate pair in Japhug (\ipa{ftʂi} vs \ipa{ndʐi}).
 
Unlike Rgyalrong languages, anticausative derivation in Stau is attested in verb with fricative initial consonants (as in \ipa{zəla}  `fall').
 
  \begin{table}[H]
 \caption{Anticausative verbs in Stau} \label{tab:anticausative} \centering 
\begin{tabular}{lllll}
\toprule
Base verb & Meaning & Anticausative verb & Meaning \\
\midrule
 \ipa{səla} &cause to fall & \ipa{zəla}  & fall \\
\ipa{pʰre}   &break (tr) &\ipa{bre}   & break (it) \\
  \ipa{fkʰe} & cut down & \ipa{vge} & break away, off \\
\ipa{ftʂə}& wake (tr) & \ipa{brə}& wake up\\
\ipa{ftɕə} &melt (tr) &\ipa{dʑə} &melt (it)\\
 \bottomrule
\end{tabular}
\end{table}
 
There is a causative \ipa{s--/z--} prefix in Stau attested in a few verbs, but unlike in Rgyalrong languages, it is not productive. Table \ref{tab:causative} provides a representative list of verb pairs. The phonological alternations attested with the \ipa{s--/z--} prefix are much less complex than those attested in Stodsde (\citealt{jackson07shangzhai}) or in Khroskyabs (\citealt{lai14caus}). However, the causative forms are not always predictable from the base form: for instance, the causative form of voiced initial verbs can be either voiced (\ipa{zgə} `put clothes on') or unvoiced ( \ipa{spərje} `burn (tr)').
 
 
   \begin{table}[H]
 \caption{Causative verbs in Stau} \label{tab:causative} \centering 
\begin{tabular}{lllll}
\toprule
Base verb & Meaning & Causative verb & Meaning \\
\midrule
 \ipa{lə}  &boil& \ipa{zɮdə}& boil (tr)\\
  \ipa{cʰu}  & hot &  \ipa{scʰu}& cook \\
    \ipa{rŋi}  & borrow &  \ipa{sŋi}& lend \\
    \ipa{qə}  & go out (fire) &  \ipa{sqʰə}& extinguish\\
    \ipa{tʰi}  & drink &  \ipa{stʰi} & give to drink\\
    \ipa{kə} / \ipa{-gə} & wear &  \ipa{zgə} & put clothes on \\
     \ipa{nə}  & burn (it) &  \ipa{snə}& burn (tr)\\
     \ipa{pərje} / \ipa{-bərje} & burn &  \ipa{spərje} & burn (tr)\\
 \bottomrule
\end{tabular}
\end{table}

Some Tibetan  loanwords, borrowed in pairs, should be distinguished from the native causative pairs (\ipa{mbjer} `be pasted on' vs \ipa{zɟwer}  `paste' from Tibetan \ipa{ɴbʲar} and \ipa{sbʲar}).
 
 
 In addition to the prefix \ipa{s-/z-}, there is one example of a causative \ipa{ɣ} prefix in the pair   \ipa{ndʑi} `learn'    $\rightarrow$  \ipa{ɣʑi} `teach'. This may be a fossilized allomorph of the causative prefix (in Khroskyabs the corresponding pair is \ipa{ndzé} `learn', \ipa{ldzê}  teach' with \ipa{l--} allomorph of the causative prefix).
     
The \ipa{s--/z--} has semantic effects that are sometimes better described with terms other than `causative'. There is one example of applicative \ipa{s--} (\ipa{qʰe}  `laugh' $\rightarrow$ \ipa{sqʰe} `laugh at') and another one of tropative \ipa{s--}  \ipa{nənʚ}  `smell (it)' $\rightarrow$ \ipa{snəsnʚ} `smell tr' (not `cause to have a smell', see \citealt{jacques13tropative}).
% zja?
 
 
incorporation

\ipa{rvatɕa} \ipa{rva}
\ipa{mbarji} `stride over' \ipa{pa}

 

\subsection{Nominalization}

\ipa{xcʰi} \ipa{ɣɟi}

 \section{Noun phrase}
\begin{table}[H]
\caption{Vowel fusion in Rtau nouns} \label{tab:alternation.noun} \centering
\begin{tabular}{c|cccccc}
\toprule
base form & meaning & ergative & genitive \\
\midrule
\ipa{kəta} & dog & \ipa{kətow} & \ipa{kətej} & \\
\ipa{vdzi} & man & \ipa{vdzu} & \ipa{vdzi} & \\
\ipa{xə} & hybrid of yak and cow & \ipa{xu} & \ipa{xi} & \\
\bottomrule
\end{tabular}
\end{table}

\begin{exe}
\ex \label{ex:rga}
\gll \ipa{tə-w}  	\ipa{ŋa-ɡi}  	\ipa{rɡa-rə}  
 \\
he-\textsc{erg} I-\textsc{dat} like-\textsc{testim} \\
\glt `(S)he likes me.'
\end{exe}

erg vs instr:


	ti ɮdow ŋa-ʁa tə-ɲcʰə (pas exprès)
	tu ɮda-kʰa ŋa-ʁa tə-ɲcʰə.
	
	
		ŋa rji-kʰa mi-mkhõ : je ne veux pas de cheval.
	ŋa rji tɕəma : je n'ai pas besoin d'un cheval

erg vs gen:

a akəstɔmbej fɕe-lə de vdẽ-ndʐə?"

focus:
xə le kʰaχcə nə-ɟi-sə ŋə-rə.
045	rkəmə le təɲə nə-ŋɔ-sə ŋərə

		tə ge vdzi ke ʁje gə ŋərə (ge: positif)
		tə le vdzi ke mtsʰer gə ŋərə (le: négatif)


topic
ŋa sə rdʑi-ʁa tɕʰu χa mi-gõ-rə 

\section{Relativization}

\section{Complementation}  

  ŋa ɲi-gi va-ɲə rɟev-ʁʚ ʁʚ
  I will help you take out the pigs


 ŋa ɕə-ba kʰã
 je dois y aller
 
 		ɮde-re ɮdi-rə : on était presque arrivé 
		ste-re ɮdã: j'ai presque fini
		ŋgə-re ɮdã: je suis prêt pour manger
		
causative:

e-ʑe nə tə-xtʂʰõ jə-rə


		sə ma-ɮdi ga tə-ntsʰu: je pense à ceux qui étaient venus.
		sə ɮdi-ga ŋa mi-dʐẽ-rə: Je ne me souviens pu de qui était venu.
\subsection{Reported speech} 
		"ədə-gi mbjoˠmbjoˠ tɕa gə tə-fkʰõ, tɕa gə tə-fkʰõ" jə-rə

 \section{TAM and evidentiality} \label{sec:TAM}

\section{Classification} \label{sec:classification}
 
 
  \citet{lai14person}


Stau \ipa{zjar} `heart', Wobzi Khroskyabs \ipa{sjɑ̂r}

Stau \ipa{mkʰə} `smoke', Wobzi Khroskyabs \ipa{mkʰə́}


\bibliographystyle{linquiry2}
\bibliography{bibliogj}
\end{document}
