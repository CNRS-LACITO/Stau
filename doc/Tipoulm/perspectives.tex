
\begin{frame}
\frametitle{Conclusion}
%\footnotesize
\begin{wideitemize}%\footnotesize
\item[\highlightii{\lefthand}]  L'approche canonique permet de fixer
  un repère par rapport auquel on peut mesurer objectivement des déviations
\item C'est un standard reposant sur la seule régularité, il est donc
  indépendant des faits observables et ne risque pas d'évoluer en
  fonction des découvertes linguistiques nouvelles
\item Il est abstrait et ne repose donc a priori pas sur un système de
  langues particulier
\item[\highlightiv{\lefthand}] Nous avons proposé une nouvelle
  définition de standard canonique pour le direct/inverse
\item Cela a permis d'évaluer qualitativement le degré de canonicité
  qu'affichent des langues particulière quant à ce standard
% \item[\highlighti{\danger}] Nous n'vons donné qu'une définition du
%   \textsc{marquage canonique du direct/inverse}
% \item Le direct/inverse en tant que mode d'encodage de l'alignement se
%   situe à l'interface entre la morphologie, la syntaxe et la sémantique
% \item Il serait à présent nécessaire d'étendre cette définition au
%   domaine syntaxique et à la théorie de l'alignement
% %\pause
\end{wideitemize}
\end{frame}

\begin{frame}
\frametitle{Perspectives}
%\footnotesize
\begin{wideitemize}%\footnotesize
% \item[\highlightii{\lefthand}]  L'approche canonique permet de fixer
%   un repère par rapport auquel on peut mesurer objectivement des déviations
% \item C'est un standard reposant sur la seule régularité, il est donc
%   indépendant des faits observables et ne risque pas d'évoluer en
%   fonction des découvertes linguistiques nouvelles
% \item Il est abstrait et ne repose donc a priori pas sur un système de
%   langues particulier
% \item[\highlightiv{\lefthand}] Nous avons proposé une nouvelle
%   définition de standard canonique pour le direct/inverse
% \item Cela a permis d'évaluer qualitativement le degré de canonicité
%   qu'affichent des langues particulière quant à ce standard
\item[\highlighti{\danger}] Nous n'avons donné qu'une définition du
  \textsc{marquage canonique du direct/inverse}
\item Le direct/inverse en tant que mode d'encodage de l'alignement se
  situe à l'interface entre la morphologie, la syntaxe et la sémantique
\item Il serait à présent nécessaire d'étendre cette définition au
  domaine syntaxique et à la théorie de l'alignement
%\pause
\end{wideitemize}
\end{frame}




