\begin{frame}
\frametitle{\darkhighlightiii{Introduction: l'inverse canonique}}
\framesubtitle{Vers une définition typologique de l'opposition direct/inverse}
%\pause
\begin{wideitemize}
\item Le direct/inverse fait partie des types de marquages de
  l'alignement
\begin{smallwideitemize}
\item un des marquages des systèmes d'alignement hiérarchique
\item autres alignements: accusatif, ergatif, neutre, tripartite
\item le plus souvent un marquage morphologique sur le verbe
\pause
\item phénomène typologique moins connu
%\pause
\end{smallwideitemize}
\end{wideitemize}
\end{frame}


\begin{frame}
\frametitle{\darkhighlightiii{Introduction: l'inverse canonique}}
\framesubtitle{La distribution des langues à marquage direct/inverse
  dans le monde}
\begin{center}

INSÉRER ICI LES CARTES AVEC LA DISTRIBUTION DES LANGUES À DIR/INV

\end{center}
\end{frame}




\begin{frame}
\frametitle{\darkhighlightiii{Introduction: l'inverse canonique}}
\framesubtitle{Vers une définition typologique de l'opposition direct/inverse}
%\pause
\begin{wideitemize}
\item Un nombre non-négligeable de langues est considé comme affichant un marquage de
  type de direct/inverse
%\begin{smallwideitemize}
\item Mais la réalisation du direct/inverse varie en fait de façon
  considérable dans ces langues\\ 
\item[\highlightii{\lefthand}] Besoin d'une approche typologique du phénomène
%\end{smallwideitemize}
\end{wideitemize}
%\item But d'une approche typologique
\begin{wideitemize}
%\begin{smallwideitemize}
\item[\highlightii{\lefthand}]~Mise en place d'un système de notions fonctionnel
\item[\highlightii{\lefthand}] approche canonique
%\end{smallwideitemize}
\end{wideitemize}
\begin{wideitemize}
\item[\highlightii{\lefthand}] But de ce papier \lefthand ~proposer une définition typologiquement
  fonctionnelle des systèmes à direct/inverse
\end{wideitemize}
\end{frame}