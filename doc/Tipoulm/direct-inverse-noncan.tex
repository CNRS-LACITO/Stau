\begin{frame}
\frametitle{Comparaison du direct/inverse à la flexion canonique}
\subtitle{La flexion canonique selon \cite{corbett07b}}
\begin{table}\centering\scriptsize
\noindent\begin{tabular}{ll@{~~~}l@{~~~}l}
\hline
&&{\sc comparison across }&{\sc
  comparison }\\
&&{\sc cells of a lexeme}&{\sc
  across lexemes}\\
\hline
1~~~&{\sc composition/structure} & {\em same}& {\em same}\\
2~~~&{\sc lexical material}  ($\approx$ shape of stem) & {\em same}& {\em different}\\
3~~~&{\sc inflectional material} ($\approx$ shape of inflection)& {\em different}& {\em same}\\
4~~~&{\sc outcome}  ($\approx$ shape of inflected word)& {\em different}& {\em different}\\\hline
&\\[-6pt]
\end{tabular}
\caption{\Small Critères définitoires de la flexion canonique selon \protect\cite{corbett07b}.}
\label{tbl:caninfl}
\end{table}
\end{frame}

\begin{frame}
\frametitle{Comparaison du direct/inverse à la flexion canonique}
\subtitle{La flexion canonique selon \cite{corbett07b}}
\begin{smallwideitemize}
\item La flexion canonique constitue un alignement parfait 1 à 1 entre une forme et un paquet de traits morphosyntaxiques (PTM)\pause ~~~\highlightii{\ding{51}}\\
\highlightii{\ra le marquage direct/inverse permet une desambiguisation totale}
\item La flexion canonique exprime chaque PTM par un affixe unique et non-ambigu \pause ~~~\highlighti{\ding{55}}\\
\highlighti{\ra abstraction faite du marquage direct/inverse toutes les marques sont ambiguë deux à deux}
\item La structure d'un paradigme canonique n'est aucunement motivé extérieurement \pause ~~~\highlighti{\ding{55}}\\
\highlighti{\ra la symmétrie dans la structure du paradigme est motivée par la hiérarchie d'empathie}

\end{smallwideitemize}

\end{frame}