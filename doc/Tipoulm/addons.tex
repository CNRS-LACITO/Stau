

\begin{frame}[noframenumbering]
\frametitle{Hétéroclise}
L'hétéroclise désigne le phénomène où le paradigme d'un lexème se
compose de parties d'au moins deux classes flexionnelles distinctes.\\[2mm]
\pause
\footnotesize
{\em Exemple: La plupart des noms d'animaux slovaques sont hétéroclites \cite{zauner73}.}\\[-2mm]
\begin{table}\small\centering
\scalebox{0.9}{
\begin{tabular}{l@{~~}|l@{~~}l@{~~}|l@{~~}l@{~~}|l@{~~}l}
\hline
&\multicolumn{2}{c@{~~~~}}{\textsc{ masc. animés}
}&
 \multicolumn{2}{c@{~~~~}}{\textsc{ 
masc. inanimés}
}
&\multicolumn{2}{c}{\textsc{ 
masc. hétéroclites}
}
\\
&\multicolumn{2}{c}{
 {\scriptsize{CHLAP}}  'gars,type'}&
 \multicolumn{2}{c}{
   {\scriptsize{DUB}}  'chêne'}
&\multicolumn{2}{c}{
 {\scriptsize{OROL}} 'aigle'}
\\
\hline
&\textsc{ singulier}&\textsc{ pluriel  }&\textsc{ singulier }&\textsc{ pluriel}&\textsc{ singulier  }&\textsc{ pluriel  }\\
\textsc{ NOM }&\cellcolor{ciel}{\em  chlap }&{\em  chlap-i}&{\em dub}&\cellcolor{mandarine}{\em dub{\bf\em -y}
}&\cellcolor{ciel}{\em orol}&\cellcolor{mandarine}{\em orl-{\bf\em y}}\\
\textsc{ GEN} &\cellcolor{ciel}{\em  chlap-a}&{\em  chlap-ov}&{\em  dub-a}&\cellcolor{mandarine}{\em dub-ov  }&\cellcolor{ciel}{\em orl-a} & \cellcolor{mandarine}{\em orl-ov}\\
\textsc{ DAT }&\cellcolor{ciel}{\em  chlap{\bf\em -ovi}}&{\em  chlap-om}&{\em
  dub-u}&\cellcolor{mandarine}{\em dub-om  }&\cellcolor{ciel}{\em orl{\bf\em -ovi}}&\cellcolor{mandarine}{\em orl-om}\\
\textsc{ ACC }&\cellcolor{ciel}{\em  chlap-a}&{\em  chlap-ov}&{\em
  dub}&\cellcolor{mandarine}{\em dub{\bf\em -y}  }&\cellcolor{ciel}{\em orl-a} & \cellcolor{mandarine}{\em orl{\bf\em -y}}\\
\textsc{ LOC} &\cellcolor{ciel}{\em  chlap{\bf\em -ovi}}&{\em  chlap-och}&{\em dub-e  }&\cellcolor{mandarine}{\em
  dub-och  }&\cellcolor{ciel}{\em orl{\bf\em -ovi}}&\cellcolor{mandarine}{\em orl-och}\\
\textsc{ INS} &\cellcolor{ciel}{\em  chlap-om}&{\em  chlap-mi}&{\em dub-om  }&\cellcolor{mandarine}{\em dub-mi  }&\cellcolor{ciel}{\em orl-om}&\cellcolor{mandarine}{\em orl-ami}\\[2ex]
\end{tabular}
}
%\caption{Heteroclisis in Slovak masculine animal names inflection}
\label{tbl:sk}
\end{table}

\end{frame}


\begin{frame}[noframenumbering]
\frametitle{Irrégularité dans le remplissage des cases}
\framesubtitle{Défectivité}
La défectivité désigne les cas où un lexème possède dans son paradigme des
cases vides (ou manquantes). 

Exemple: {\em Activa tantum} (latin) \cite{kiparsky04}: verbes de
remplacement pour combler le trou dans le sous-paradigme passif.

\noindent\eenumsentence{%\footnotesize
\item \shortex{3}{ {\scriptsize{FACIO}} & \pause $\longrightarrow$ &
    \pause  {\scriptsize{FIO}} }{``faire''&&\pause ``faire \textsc{(pass)}''}{}\pause
\item \shortex{3}{ {\scriptsize{PERDIO}}  & $\longrightarrow$ &  {\scriptsize{PEREO}}}{``détruire''&&
    ``détruire \textsc{(pass)}''}{}
}

\pause
\begin{itemize}
\item Les substituts ne
      fonctionnent pas uniquement comme formes passives de  {\scriptsize{FACIO}} 
    et {\scriptsize{PERDIO}}.
  \item Ils sont également des verbes transitifs
    indépendants à part entière.
  \item Ils constituent des entrées lexicales
    indépendantes.
\end{itemize}
\end{frame}



\begin{frame}[noframenumbering]
\frametitle{Irrégularité dans le remplissage des cases}
\framesubtitle{Surabondance en français}

\begin{itemize}
\item  {\em Asseoir} possède
deux formes distinctes dans la majorité de ses cases (Cf.~\cite{bonami10}).
\item Tous les verbes français en {\em
  -ayer} affichent une surabondance systématique.
\end{itemize}

\pause
\begin{table}\centering
\scalebox{0.9}{
\begin{tabular}{ccc}
    \begin{minipage}[b]{.4\linewidth}\centering\small

\scalebox{0.9}{
\begin{tabular}{l@{~~~}l@{~~~}l}
\hline
\textsc{ ind.pres}  & \textsc{ singulier} & \textsc{ pluriel}\\
\hline
\multirow{2}{*}\textsc{ p1}&  {\em assois}& {\em assoyons}\\
&  {\em assieds}& {\em asseyons}\\[1ex]
\multirow{2}{*}\textsc{ p2}& {\em assois}& {\em assoyez}\\
&  {\em assieds}& {\em asseyez}\\[1ex]
\multirow{2}{*}\textsc{ p3}&  {\em assoit}& {\em assoient}\\
&  {\em assied}& {\em asseyent}\\
\end{tabular}
}
\caption{Surabondance en français pour le verbe
  {\em asseoir}}
\label{tbl:asseoir}
\end{minipage}
&~~~~~~~~&

    \begin{minipage}[b]{.4\linewidth}\centering\small
\scalebox{0.9}{
\begin{tabular}{l@{~~~}l@{~~~}l}
\hline
\textsc{ ind.pres}  & \textsc{ singulier} & \textsc{ pluriel}\\
\hline
\multirow{2}{*}\textsc{ p1}&  {\em balaye}& \multirow{2}{*}{\em balayons}\\
&  {\em balaie}& {\em }\\[1ex]
\multirow{2}{*}\textsc{ p2}& {\em balayes}& \multirow{2}{*}{\em balayez}\\
&  {\em balaies}& {\em }\\[1ex]
\multirow{2}{*}\textsc{ p3}&  {\em balaye}& {\em balayent}\\
&  {\em balaie}& {\em balaient}\\
\end{tabular}
}
\caption{Surabondance en français pour le verbe
  {\em balayer}}
\label{tbl:balayer}
\end{minipage}
\end{tabular}
}
\end{table}
\end{frame}
