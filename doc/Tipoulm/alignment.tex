\subsection[Alignements]{Types d'alignements à travers les langues du monde}

% a) Types d'alignements: cas idéalisés (G): ACC, ERG, NEUTRE, TRIPARTITE, HIERARCHIQUE

% \& exemples: 
% ACC - français (G); 
% ERG - basque (A); 
% NEUTRE - vietnamien (G); 
% TRIPARTITE - thulung (G); 
% HIERARCHIQUE - dargwa (G)


\begin{frame}
\frametitle{Types d'alignement}
\framesubtitle{L'alignement accusatif (cas idéalisé)}
% encodage morphologique au sein du système nominal: latin (g)


\end{frame}


\begin{frame}
\frametitle{Types d'alignement}
\framesubtitle{L'alignement accusatif (français)}
% encodage morphologique au sein du système nominal: latin (g)


\end{frame}


\begin{frame}
\frametitle{Types d'alignement}
\framesubtitle{L'alignement ergatif (cas idéalisé)}
% encodage morphologique au sein du système nominal: latin (g)


\end{frame}


\begin{frame}
\frametitle{Types d'alignement}
\framesubtitle{L'alignement ergatif (basque)}
% encodage morphologique au sein du système nominal: latin (g)


\end{frame}


\begin{frame}
\frametitle{Types d'alignement}
\framesubtitle{L'alignement neutre (cas idéalisé)}
% encodage morphologique au sein du système nominal: latin (g)


\end{frame}



\begin{frame}
\frametitle{Types d'alignement}
\framesubtitle{L'alignement neutre (vietnamien)}
% encodage morphologique au sein du système nominal: latin (g)


\end{frame}


\begin{frame}
\frametitle{Types d'alignement}
\framesubtitle{L'alignement tripartite (cas idéalisé)}
% encodage morphologique au sein du système nominal: latin (g)


\end{frame}

\begin{frame}
\frametitle{Types d'alignement}
\framesubtitle{L'alignement tripartite (thulung)}
% encodage morphologique au sein du système nominal: latin (g)


\end{frame}

\subsection[Encodage des alignements]{Encodage des alignements à travers les langues du monde}
% b) encodage des alignements (G)
% encodage syntaxique par l'ordre des mots: anglais (G)
% encodage syntaxique par des mots autonomes (préposisitons): mandarin (G)

\begin{frame}
\frametitle{Encodage syntaxique}
\framesubtitle{Ordre des mots (anglais)}
% encodage morphologique au sein du système nominal: latin (g)

\ex. Ashley love-s Bradley.\\
Ashley aimer\textsc{-3sg} Bradley\\
{\em ``Ashley aime Bradley.''}

\ex. Bradley love-s Ashley.\\
Bradley aimer\textsc{-3sg} Ashley\\
{\em ``Bradley aime Ashley.''}

\end{frame}

\begin{frame}
\frametitle{Encodage syntaxique}
\framesubtitle{Marqueurs syntaxiques (chinois mandarin)}
% encodage morphologique au sein du système nominal: latin (g)




\end{frame}


\begin{frame}
\frametitle{Encodage morphologique}
\framesubtitle{Morphologie nominale (latin)}
% encodage morphologique au sein du système nominal: latin (g)

\ex. Agricol-a lup-um vide-t.\\
paysan\textsc{(m)-nom.sg} loup\textsc{(m)-acc.sg} voir\textsc{$\backslash$\si}.-\textsc{3sg.ind.prs.act}\\
{\em ``Le paysan voit le loup.''}

\ex. Agricol-am lup-us vide-t.\\
paysan\textsc{(m)-acc.sg} loup\textsc{(m)-nom.sg} voir\textsc{$\backslash$\si}.-\textsc{3sg.ind.prs.act}\\
{\em ``Le loup voit le paysan.''}

\end{frame}

\begin{frame}
\frametitle{Encodage morphologique}
\framesubtitle{Morphologie verbale (kurde sorani)}
% encodage morphologique au sein du système verbal: kurde sorani (g)

\ex. xward-in\alert{=î} \\
eat$\backslash$\sii-{\sc 3pl}=\textsc{3sg}\\
{\em ``Il les a mangé(e)s.''}

\end{frame}

\iffalse
\subsection[Bantawa]{Un exemple de paradigme: le bantawa}
% c) exemple du Bantawa comme cas incompréhensible (G)


\begin{frame}
\frametitle{Exemple de paradigme}
\framesubtitle{Sous-paradigme affirmatif non-passé du bantawa (kiranti))}
\begin{table}[H]
\caption{The Bantawa non-past affirmative transitive paradigm (\cite[145-8]{doornenbal09})}\label{tab:bantawapos}
\resizebox{\textwidth}{!}{
\begin{tabular}{llllllllllll}
%\cline{1-12}
\toprule
\backslashbox{A}{P}  & 	\textsc{1sg} & 	\textsc{1di} & 	\textsc{1de} & 	\textsc{1pi} & 	\textsc{1pe} & 	\textsc{2sg} & 	\textsc{2du} & 	\textsc{2pl} & 	\textsc{3sg} & 	\textsc{3du} & \textsc{3pl}\\
% \cline{1-1}
% \cline{7-12}
\midrule
\textsc{1sg} &  \multicolumn{5}{c}{\grise{}}	 	&	\ipa{\ro{}-na} & 	\ipa{\ro{}-naci} & 	\ipa{\ro{}-nanin}& 	\ipa{\ro{}-uŋ}\cellcolor[wave]{450} & 	\multicolumn{2}{c}{\ipa{\ro{}-uŋc{\textbari}ŋ}\cellcolor[wave]{450}} 	\\
\textsc{1di} & \multicolumn{8}{c}{\grise{}}	 		&	\ipa{\ro{}-cu}\cellcolor[wave]{450} & 	\multicolumn{2}{c}{\ipa{\ro{}-cuci}\cellcolor[wave]{450}}	\\
\textsc{1de} & 	 \multicolumn{5}{c}{\grise{}}	&	 	 \multicolumn{3}{c}{\ipa{\ro{}-ni}}     & 	\ipa{\ro{}-cu{\textglotstop}a}\cellcolor[wave]{450} & 	\multicolumn{2}{c}{\ipa{\ro{}-cuci{\textglotstop}a}\cellcolor[wave]{450}} 	\\
\textsc{1pi} & 	 \multicolumn{8}{c}{\grise{}}	&	\ipa{\ro{}-um}\cellcolor[wave]{450} & \multicolumn{2}{c}{	\ipa{\ro{}-umc{\textbari}m}\cellcolor[wave]{450}} 	\\
\textsc{1pe} & 	 \multicolumn{5}{c}{\grise{}}&		 \multicolumn{3}{c}{\ipa{\ro{}-ni}} & 	\ipa{\ro{}-umka}\cellcolor[wave]{450} & 	\multicolumn{2}{c}{\ipa{\ro{}-umc{\textbari}mka}\cellcolor[wave]{450}} 	\\
%\cline{10-12}
\textsc{2sg} & 	\ipa{t{\textbari}-\ro{}-ŋa} & 	\grise{}&	\ipa{} & \grise{}	&	\ipa{} & 	 \multicolumn{3}{c}{\grise{}}	&	\ipa{t{\textbari}-\ro{}-u}\cellcolor[wave]{450} & 	\multicolumn{2}{c}{\ipa{t{\textbari}-\ro{}-uci}\cellcolor[wave]{450}} \\
\textsc{2du} & 	\ipa{t{\textbari}-\ro{}-ŋaŋc{\textbari}ŋ} & \grise{}&	\ipa{t{\textbari}-\ro{}-ni(n)} & \grise{}	&	\ipa{t{\textbari}-\ro{}-ni(n)} & 	 \multicolumn{3}{c}{\grise{}}	&	\ipa{t{\textbari}-\ro{}-cu}\cellcolor[wave]{450} & 	\multicolumn{2}{c}{\ipa{t{\textbari}-\ro{}-cuci}\cellcolor[wave]{450}}\\
\textsc{2pl} & 	\ipa{t{\textbari}-\ro{}-ŋaŋn{\textbari}ŋ} & \grise{}	&	\ipa{} & \grise{}	&	\ipa{} & 	 \multicolumn{3}{c}{\grise{}}&	\ipa{t{\textbari}-\ro{}-um}\cellcolor[wave]{450} & 	\multicolumn{2}{c}{\ipa{t{\textbari}-\ro{}-umcum}\cellcolor[wave]{450}} \\
%\cline{2-12}
\textsc{3sg} &\cellcolor[wave]{650}	 	\ipa{{\textbari}-\ro{}-ŋa} & 	 \cellcolor[wave]{650}	 & 	\cellcolor[wave]{650}	\ipa{(n){\textbari}-\ro{}-aci{\textglotstop}a} &    \cellcolor{red}	& \cellcolor[wave]{650}		\ipa{(n){\textbari}-\ro{}-inka} & 	\cellcolor[wave]{650}	 & 	\cellcolor[wave]{650}	& 	\cellcolor[wave]{650}	& 	\ipa{\ro{}-u}\cellcolor[wave]{450} & 	\multicolumn{2}{c}{\ipa{\ro{}-uci}\cellcolor[wave]{450}} \\
\textsc{3du} &\ipa{{\textbari}-\ro{}-ŋaŋc{\textbari}ŋ}\cellcolor[wave]{650} & 	  \ipa{n{\textbari}-\ro{}-ci}\cellcolor[wave]{650} 	& 	\cellcolor[wave]{650}{\multirow{2}{*}{\ipa{n{\textbari}-\ro{}-aci{\textglotstop}a}}}	 & 	 \ipa{m{\textbari}-\ro{}}\cellcolor{red} 	 & \multirow{2}{*}{\ipa{n{\textbari}-\ro{}-inka}\cellcolor[wave]{650}} & 	\cellcolor[wave]{650}	\ipa{n{\textbari}-\ro{}} & \ipa{n{\textbari}-\ro{}-ci}\cellcolor[wave]{650} & 	\ipa{n{\textbari}-\ro{}-in}\cellcolor[wave]{650} & \ipa{{\textbari}-\ro{}-cu} \cellcolor[wave]{650}& \multicolumn{2}{c}{\ipa{{\textbari}-\ro{}-cuci}\cellcolor[wave]{650}}	\\
\textsc{3pl} &	 \ipa{n{\textbari}-\ro{}-ŋa}\cellcolor[wave]{650} & \cellcolor[wave]{650}	 &  \multirow{-2}{*}{\ipa{n{\textbari}-\ro{}-aci{\textglotstop}a}\cellcolor[wave]{650}}	  & 	\cellcolor{red} &  \multirow{-2}{*}{\ipa{n{\textbari}-\ro{}-inka}\cellcolor[wave]{650}}	 &   \cellcolor[wave]{650}		  & 	 \cellcolor[wave]{650}	 & \cellcolor[wave]{650}	   & 	\ipa{{\textbari}-\ro{}} \cellcolor[wave]{650}	& \multicolumn{2}{c}{\ipa{m{\textbari}-\ro{}-uci}\cellcolor[wave]{650}} 	\\
%\cline{1-12}
\bottomrule
\textsc{intr}	&\ipa{\ro{}-ŋa}&\ipa{\ro{}-ci}&\ipa{\ro{}-ca}&\ipa{\ro{}-in}&\ipa{\ro{}-inka}&\ipa{t{\textbari}-\ro{}}& \ipa{t{\textbari}-\ro{}-ci}& \ipa{t{\textbari}-\ro{}-in}& \ipa{\ro{}}  & \ipa{\ro{}-ci} &\ipa{m{\textbari}-\ro{}} \\

\bottomrule
\end{tabular}}
\end{table}

\end{frame}

\begin{frame}
\frametitle{Exemple de paradigme}
\framesubtitle{Sous-paradigme négatif non-passé du bantawa
  (kiranti))}
\begin{table}[H]
\caption{The Bantawa non-past negative transitive paradigm (\cite[145-8]{doornenbal09})}\label{tab:bantawaneg}
\resizebox{\textwidth}{!}{
\begin{tabular}{llllllllllll}
\toprule
\backslashbox{A}{P}  & 	\textsc{1sg} & 	\textsc{1di} & 	\textsc{1de} & 	\textsc{1pi} & 	\textsc{1pe} & 	\textsc{2sg} & 	\textsc{2du} & 	\textsc{2pl} & 	\textsc{3sg} & 	\textsc{3du} & \textsc{3pl}	\\
%\cline{1-1}
%\cline{7-12}
\midrule
\textsc{1sg} &  \multicolumn{5}{c}{\grise{}}	 	&	{\ipa{{\textbari}-\ro{}-nan}} & 	\ipa{{\textbari}-\ro{}-nancin} & 	\ipa{{\textbari}-\ro{}-naminin}& 	\cellcolor{red}\ipa{{\textbari}-\ro{}-n{\textbari}ŋ}& 	\multicolumn{2}{c}{\ipa{{\textbari}-\ro{}-n{\textbari}ŋc{\textbari}ŋ}\cellcolor{red}} 	\\
\textsc{1di} & \multicolumn{8}{c}{\grise{}}	 		&	\cellcolor{red}\ipa{{\textbari}-\ro{}-cun} & 	\multicolumn{2}{c}{\ipa{{\textbari}-\ro{}-cuncin}\cellcolor{red}}	\\
\textsc{1de} & 	 \multicolumn{5}{c}{\grise{}}	&	 	 \multicolumn{3}{c}{\ipa{{\textbari}-\ro{}-nin}}     & 	\cellcolor{red}\ipa{{\textbari}-\ro{}-cunka}& 	\multicolumn{2}{c}{\ipa{{\textbari}-\ro{}-cuncinka}\cellcolor{red}} 	\\
\textsc{1pi} & 	 \multicolumn{8}{c}{\grise{}}	&	\cellcolor{red}\ipa{{\textbari}-\ro{}-imin}& \multicolumn{2}{c}{\ipa{{\textbari}-\ro{}-imincin}\cellcolor{red}} 	\\
\textsc{1pe} & 	 \multicolumn{5}{c}{\grise{}}&		 \multicolumn{3}{c}{\ipa{{\textbari}-\ro{}-nin}} & 	\cellcolor{red}\ipa{{\textbari}-\ro{}-iminka} & 	\multicolumn{2}{c}{\ipa{{\textbari}-\ro{}-imincinka}\cellcolor{red}} 	\\
%\cline{10-12}
\textsc{2sg} & 	\ipa{t{\textbari}-\ro{}-n{\textbari}ŋ} & 	\grise{}&	\ipa{} & \grise{}	&	\ipa{} & 	 \multicolumn{3}{c}{\grise{}}	&	\ipa{t{\textbari}-\ro{}-nan} & 	\multicolumn{2}{c}{\ipa{t{\textbari}-\ro{}-nancin}} \\
\textsc{2du} & 	\ipa{t{\textbari}-\ro{}-ŋ{\textbari}ŋc{\textbari}ŋ} & \grise{}&	\ipa{t{\textbari}-\ro{}-niminin} & \grise{}	&	\ipa{t{\textbari}-\ro{}-niminin} & 	 \multicolumn{3}{c}{\grise{}}	&	\ipa{t{\textbari}-\ro{}-nancin} & 	\multicolumn{2}{c}{\ipa{t{\textbari}-\ro{}-nancinan}}\\
\textsc{2pl} & 	\ipa{t{\textbari}-\ro{}-ŋ{\textbari}ŋm{\textbari}n{\textbari}ŋ} & \grise{}	&	\ipa{} & \grise{}	&	\ipa{} & 	 \multicolumn{3}{c}{\grise{}}&	\ipa{t{\textbari}-\ro{}-naminin}& 	\multicolumn{2}{c}{\ipa{t{\textbari}-\ro{}-nannimincin}} \\
%\cline{2-12}
\textsc{3sg} & \ipa{{\textbari}-\ro{}-n{\textbari}ŋ} & 	& &  &  & & & & \ipa{{\textbari}-\ro{}-un} & 	\multicolumn{2}{c}{\ipa{{\textbari}-\ro{}-uncin}} \\
\textsc{3du} & \ipa{{\textbari}-\ro{}-ŋ{\textbari}ŋc{\textbari}ŋ} &   \ipa{n{\textbari}-\ro{}-cin} 	& 	 \ipa{n{\textbari}-\ro{}-cinka}	 & 	 \ipa{m{\textbari}-\ro{}-nin} 	 &	\ipa{n{\textbari}-\ro{}-iminka} & 	\ipa{n{\textbari}-\ro{}-nan} & 	\ipa{n{\textbari}-\ro{}-nancin} & 		\ipa{n{\textbari}-\ro{}-naminin} & \ipa{{\textbari}-\ro{}-cun}& \multicolumn{2}{c}{\ipa{{\textbari}-\ro{}-cuncin}}	\\
\textsc{3pl} & 	\ipa{n{\textbari}-\ro{}-n{\textbari}ŋ} &  	  & 	  & 	 & 	 & 	  & 	 	  & 	   & 	\ipa{n{\textbari}-\ro{}-un} 	& \multicolumn{2}{c}{\ipa{n{\textbari}-\ro{}-uncin}} 	\\
%\cline{1-12}
\bottomrule
\textsc{intr}	&\cellcolor{red}\ipa{{\textbari}-\ro{}-n{\textbari}ŋ}&\cellcolor{red}\ipa{{\textbari}-\ro{}-cin}&\cellcolor{red}\ipa{{\textbari}-\ro{}-cinka}&\cellcolor{red}\ipa{{\textbari}-\ro{}-imin}&\cellcolor{red}\ipa{{\textbari}-\ro{}-iminka}&\ipa{t{\textbari}-\ro{}-nan}& \ipa{t{\textbari}-\ro{}-nanci}& \ipa{t{\textbari}-\ro{}-naminin}& \cellcolor{red}\ipa{{\textbari}-\ro{}-nin}  & \cellcolor{red}\ipa{{\textbari}-\ro{}-cin} &\ipa{n{\textbari}-\ro{}-nin} \\
\bottomrule
\end{tabular}}
\end{table}
\end{frame}

\fi